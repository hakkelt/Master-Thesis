\chapter{Introduction}

While the fast evolution of technology profoundly changed today's medicine, unarguably the medical imaging is of the fields which profited the most of the computation power recently became available. And as X-ray radiographs revolutionized medical treatments in the beginning of the 20th century,  the appearance of computer-aided imaging techniques such as computer tomography (CT), diagnostic ultrasonography, positron emisson tomography (PET), and magnetic resonance imaging (MRI) opened a new horizon drastically increasing the resolution, allowing 3D imaging, providing reliable dynamic recordings, and enhancing images by automated post-processing. In the recent decades radiology evolved to be an interdisciplinary field involving, for instance, molecular biology, nuclear physics, applied mathematics, and computer science besides the classical medical fields such as anatomy, angiology, and cardiology.

\section{Magnetic Resonance Imaging}

\subsection{History of Medical Imaging}
The beginning of the history of medical images dates back to November of 1895 when Wilhelm Conrad Röntgen discovered X-rays. Besides the remarkable fact that he was awarded the first Nobel Prize in Physics in 1901, it also signifies the importance of his discovery that up until the 1960's it was the only medical imaging technique available. During the first 60 years of X-ray radiography, it underwent a remarkable development gradually increasing the resolution of images and decreasing the radiation dose that affected both the patients' and the doctors' health. Since the 1920s, visualization of motions (i.e. dynamic imaging) became possible by flouroscopy, although only to a limited extent~\cite{bradley_history_2008}.

While the theoretical background of later advances in medical images were present much earlier, the the introduction to clinical medicine took place only when the computers became powerful enough for image reconstruction tasks concerning both the computational power and the available memory. Therefore, the major breakthrough came only in the 1960-70s, when suddenly multiple imaging techniques were introduced. The first of these methods was the ultrasound imaging devised by Floyd Firestone in 1940 to detect internal flaws in metal castings~\cite{singh_origin_2007}, and used first for medical purposes in 1949 by John Wild~\cite{watts_john_2009}, but it was not until 1961 when David Robinson and George Kossoff developed the first commercially practical water path ultrasonic scanner~\cite{griffiths_historical_nodate}. Also, the concept of emission and transmission tomography was propsed in the late 1950s by David E. Kuhl, Luke Chapman and Roy Edwards, but computer tomography was invented in 1972 by Godfrey Hounsfield and Allan Cormack~\cite{richmond_sir_2004}, and the first PET camera was built for human studies by Edward Hoffman, Michael M. Ter-Pogossian, and Michael E. Phelps in 1973~\cite{noauthor_us_nodate}. Moreover, nuclear magnetic resonance (NMR) imaging was discovered by Felix Bloch and Edward Purcell in 1946, and Reymond Vahan Damadian proposed the first MR body scanner in 1969. Then Paul Lauterbur had the idea of applying magnetic field gradients in all three dimensions and a back-projection technique to create images in 1971, and he also published the first MRI images: water tubes, a living clam, and the thoracic cavity of a mouse in 1973 and 1974~\cite{rinck_short_2008}. Since then, these computer-aided methods continue to develop at a very fast pace, becoming essential tools for today's practitioners.

\subsection{Applications in Diagnostics}
Even though MRI is a relative new technology, it has already revolutionized medical imaging and diagnostic process as we know it. Its versatility makes it fit a wide range of use cases. Compared to other imaging technologies, MRI demonstrates important advantages in many cases. 
\begin{itemize}
    \item In contrast to X-ray, MRI doesn't use any ionizing radiation, and hence it is totally harmless to the patient. Also, MRI is much better at imaging soft tissues, in particular neural tissue,  while X-rays are rather used for diagnosing bone degeneration, dislocation, fracture, tumor and infection. Furthermore, MRI allows 3D scans. On the other hand, X-ray machines are quite inexpensive compared to MRI, and acquires the image multiple times faster.
    \item As CT scanning is also based on X-rays, it shares this downside with X-rays, doctors need to evaluate the possible benefits of the scan and decide if it outweighs the potential complications of exposure to ionizing radiations. MRI, however, elicit this problem. although at the price of a elongated imaging process. Comparing the medical problems where these technologies are used, one can conclude that CT scan is very helpful in diagnosing severe injuries of the chest, head, spine or abdomen, particularly fractures, and it is commonly used to localize tumors. An MRI, however, often performs better at diagnosing problems in the joints, soft tissues, ligaments and tendons. Doctors use it frequently to scan the spine, brain, muscles, neck, breasts, and abdomen.
    \item The strongest point of sonography is its portability, low cost, and real-time imaging speed without any harmful radiation, but this technology is rather limited to 2D imaging (although 3D imaging is possible), have trouble penetrating bone, and even in absence of bone the depth of penetration is limited depending on the frequency of imaging.
    \item PET scans are particularly useful for functional imaging; for instance, in identification of lapses in cognitive function, examination of cardiac failures, cancer screening and diagnosis, and finding an infection. The main disadvantage of this technology is nevertheless that the acquisition is relatively long (especially, if we consider also the time while patients wait for the tracer to reach the targeted organ), it uses a radioactive substance as tracer, and it cannot scan tissues not absorbing the tracer making the localization of the source of the signal challenging when no additional information is available. In practice, however, the letter limitation is solved by combining PET scanners with either CT or MRI.
\end{itemize}
To sum up, MRI is a strong competitor to other imaging technologies, but it also have weaknesses, of which the slow acquisition time tend to be the most problematic. There are many methods to speed up measurements as it will be discussed later, but the construction cost and the hardware constraints limit the applicability of these efforts. The problem of slowness is even more apparent in case of dynamic images as motion of organs (e.g. heart or lung) can drastically degrade the image quality. To overcome that issue, software solutions get more and more attention, especially since the appearance of compressed sensing based reconstruction methods.

\section{Compressed Sensing}
\subsection{Core Concept}
\subsection{Application Areas}
\subsection{Suitability to MRI}

\section{Julia Language}
\subsection{Objectives of The Language}
\subsection{Suitability to Our Task}

\section{Objective}

% ----------------------------------------------------
\section{Outline}
The summary of the chapters of the thesis work:

\paragraph{Chapter 2} This chapter describes something and here I summarize it in a couple sentences.

\paragraph{Chapter 3} This chapter describes something and here I summarize it in a couple sentences.

\paragraph{Chapter 4} This chapter describes something and here I summarize it in a couple sentences.

\paragraph{Chapter 5} This chapter describes something and here I summarize it in a couple sentences.

\clearpage % You need \clearpage at the end of every chapter to force images included in this chapter to be rendered in somewhere else