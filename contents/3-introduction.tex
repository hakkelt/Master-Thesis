\chapter{Introduction}\label{chapter:introduction}

While the fast evolution of technology profoundly changed today's medicine, unarguably the medical imaging is of the fields which profited the most of the computation power recently became available. And as X-ray radiographs revolutionized medical treatments in the beginning of the 20th century,  the appearance of computer-aided imaging techniques such as computer tomography (CT), diagnostic ultrasonography, positron emission tomography (PET), and magnetic resonance imaging (MRI) opened a new horizon drastically increasing the resolution, allowing 3D imaging, providing reliable dynamic recordings, and enhancing images by automated post-processing. In the recent decades radiology evolved to be an interdisciplinary field involving, for instance, molecular biology, nuclear physics, applied mathematics, and computer science besides the classical medical fields such as anatomy, angiology, and cardiology.

\section{Magnetic Resonance Imaging}

In particular, MRI has revolutionized medical imaging and diagnostic process as we know it. Its versatility makes it fit a wide range of use cases. Compared to other imaging technologies, MRI demonstrates important advantages in many cases.
\begin{itemize}
    \item In contrast to X-ray, MRI doesn't use any ionizing radiation, and hence it is totally harmless to the patient. Also, MRI is much better at imaging soft tissues, in particular neural tissue,  while X-rays are rather used for diagnosing bone degeneration, dislocation, fracture, tumor and infection. Furthermore, MRI allows 3D scans. On the other hand, X-ray machines are quite inexpensive compared to MRI, and acquires the image multiple times faster.
    \item As CT scanning is also based on X-rays, it shares this downside with X-rays, doctors need to evaluate the possible benefits of the scan and decide if it outweighs the potential complications of exposure to ionizing radiations. MRI, however, elicit this problem. although at the price of a elongated imaging process. Comparing the medical problems where these technologies are used, one can conclude that CT scan is very helpful in diagnosing severe injuries of the chest, head, spine or abdomen, particularly fractures, and it is commonly used to localize tumors. An MRI, however, often performs better at diagnosing problems in the joints, soft tissues, ligaments and tendons. Doctors use it frequently to scan the spine, brain, muscles, neck, breasts, and abdomen.
    \item The strongest point of sonography is its portability, low cost, and real-time imaging speed without any harmful radiation, but this technology is rather limited to 2D imaging (although 3D imaging is possible), have trouble penetrating bone, and even in absence of bone the depth of penetration is limited depending on the frequency of imaging.
    \item PET scans are particularly useful for functional imaging. For instance, it is used for identification of lapses in cognitive function, examination of cardiac failures, cancer screening and diagnosis, and finding an infection. The main disadvantage of this technology is nevertheless that image acquisition is relatively long (especially, if we consider also the time while patients wait for the tracer to reach the targeted organ), it uses a radioactive substance as tracer, and it cannot scan tissues not absorbing the tracer making the localization of the source of the signal infeasible without any additional information. In practice, however, the latter limitation is solved by combining PET scanners with either CT or MRI.
\end{itemize}
To sum up, MRI is a strong competitor to other imaging technologies, but it also have weaknesses, of which the slow scanning time tend to be the most problematic. There are many methods to speed up measurements as it will be discussed later, but the construction cost and the hardware constraints limit the applicability of these efforts. The problem of slowness is even more apparent in case of dynamic images as motion of organs (e.g. heart or lung) can drastically degrade the image quality. To overcome that issue, software solutions get more and more attention.

\section{Compressed Sensing}

Among the many software techniques invented to improve image quality, compressed sensing (also known as compressive sensing, compressed sampling, and compressive sampling) is inevitably is one of the most impactful theoretical construction, introduced by Donoho, Candès, Romberg, and Tao in 2004~\cite{candes_robust_2006, donoho_compressed_2006, candes_nearoptimal_2006}. In contrast to the Nyquist-Shannon theorem that asserts that continuous band-limited signals can be perfectly reconstructed from samples taken at a rate of twice the highest frequency present in the signal of interest, compressed sensing allows lossless reconstruction from much lower number of samples under certain conditions.

This impressive improvement is due to the same phenomenon that makes modern image compression algorithms so successful: the sparsity of the signal to be recovered in a certain transform domain. And while the classic image processing flow starts with acquiring the fully sampled image, then feeding it to a compression algorithm that discards the vast majority of the data still allowing later a lossless decompression, the idea behind compressed sensing is that image acquisition can be made much more effective by fusing it with the compression step recording only the data we need later for decompression, hence the name compressed sensing. Since MRI scanning operates directly in Fourier domain and all natural images tent to be sparse in the Fourier domain, compressed sensing is particularly effective in accelerating MRI acquisition.

The only drawback of using compressed sensing, however, is that the process of reconstructing the image from the measured data is much more complicated than from the fully sampled measurement. Therefore, a large number of studies investigates the possible solutions for the recovery of a high resolution image from the compressed representation. Most of these attempts starts from an already existing optimization method, defines a cost function which is hoped to lead to a more optimal solution, and maybe combines the resulted algorithm with some extra steps helping faster convergence or more exact recovery. But at the same time, new optimization algorithms or variants of existing algorithms are developed continuously, so another way to approach the problem is to try out new algorithms within the conventional frameworks.

\section{Julia Language}
\subsection{Objectives of The Language}
\subsection{Suitability to Our Task}

\section{Objective}

% ----------------------------------------------------
\section{Outline}
The summary of the chapters of the thesis work:

\paragraph{Chapter 2} This chapter describes something and here I summarize it in a couple sentences.

\paragraph{Chapter 3} This chapter describes something and here I summarize it in a couple sentences.

\paragraph{Chapter 4} This chapter describes something and here I summarize it in a couple sentences.

\paragraph{Chapter 5} This chapter describes something and here I summarize it in a couple sentences.

\clearpage % You need \clearpage at the end of every chapter to force images included in this chapter to be rendered in somewhere else