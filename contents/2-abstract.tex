\chapter*{Abstract}

Although MR imagig was invented less then 50 years ago, it has revolutionized medical imaging and diagnostic process as we know it. Its versatility makes it fit a wide range of use cases, and compared to other imaging technologies, MRI demonstrates important advantages in many cases, such as lack of ionizing radiation, adjustable contrast range, and excellent differentiation between soft tissues. However, it has also a disadvantage: the imaging speed is undesirably slow. While it is only inconvenience in some cases, dynamic images of chest, for example, gets blurred by the different motions of the body, like breathing and cardiac motion. There are new technologies that improves the imaging process itself, but strong image reconstruction algorithms can vastly improve the quality.

Among the many software techniques invented to improve image quality, compressed sensing is inevitably is one of the most impactful theoretical construction, introduced by Donoho, Candès, Romberg, and Tao in 2004~\cite{candes_robust_2006, donoho_compressed_2006, candes_nearoptimal_2006}. In contrast to the Nyquist-Shannon theorem that asserts that continuous band-limited signals can be perfectly reconstructed from samples taken at a rate of twice the highest frequency present in the signal of interest, compressed sensing allows lossless reconstruction from much lower number of samples given that certain natural conditions are satisfied.

In this thesis work, we consider the classic results as well as the recent advances within of the compressed sensing framework and their application to real life MR imaging. In particular, we closely examine two recent publications presenting state-of-the-art solutions combining conventional techniques with novel ideas.

Afterwards, we present our implementation of these algorithms along with the implementation of a recently invented algorithm from the family of iteratively reweighted least squares (IRLS) methods that previously have not been applied to MRI setting yet. For the language of implementation we have chosen Julia, a new open-source language released in 2012, as this language fits well the image reconstruction problem being inherently fast with a speed often comparable to C, and providing convenient environment for fast prototyping.

Finally, we compare these algorithm with respect to reconstruction power from massively undersampled data and noise tolerance. Our results demonstrates the fitness of the Julia language to prototyping and also shows strong benchmark suggesting that a robust extension of the new IRLS method can bring enormous improvement to reconstruction quality or imaging speed.

\clearpage