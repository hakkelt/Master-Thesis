%% In this file, the package imports are listed, style and formatting rules are declared, and in the last couple rows the contents.tex is included, thus this file is the entry point for rendering.

% General rules
% --------------------------------------------------

% twoside: printing margins are set two sided printing --> you can change it to "oneside" (but, please, protect environment and don't do that especially because the "Code of Studies and Exams" of the Faculty also recommends printing two sided)
% 11pt: font size
\documentclass[a4paper,11pt,twoside]{report}

% Uncomment this line if you want to change the default font family to palatino:
\renewcommand{\familydefault}{ppl}
% Here you can browse more fonts supported by Overleaf (just make sure to use a serif font typeface): https://www.sharelatex.com/learn/Font_typefaces

% Package needed to format table of contents (ToC):
\usepackage{tocloft}
% Indentations of sections in ToC:
\setlength{\cftsecindent}{.9cm}
% Indentations of subsections in ToC:
\setlength{\cftsubsecindent}{1.4cm}

% Uncomment the following line if you want the renderer insert space between paragraphs separated by an empty line
%\usepackage{parskip}
% ---------------------------------------------------

% Formatting chapter titles
% ---------------------------------------------------
\usepackage{titlesec}

% Declare the depth of numbering of chapters/sections/subsections:
% 0: only chapters are numbered
% 1: chapters and sections are numbered
% 2: even subsections are numbered
\setcounter{secnumdepth}{2} 

% Format text of chapter titles
% Uncommenting the line within the braces will change the chapter titles from "
%   Chapter {number}
%   {title}
% " to "
%   {number}. {title}
% "
\newcommand{\titleformatting}{
    %\titleformat{\chapter}{\normalfont\huge}{\thechapter.}{1em}{\huge\textbf}
}

% Uncommenting the following line will remove the number of the chapter before the number of the subsections (e.g. changes "2.1 {section title}" to "1. {section title}" where "2" is the number of the chapter, and "1" is the number of the section
%\renewcommand\thesection{\arabic{section}}

% Format text of section titles
%\titleformat{<which element you want to format>}[block/wrap]{<format rules for numbers of sections>}{<commands that render the numbers>}{<how much space you want between the number and the title>}{<format rules for title>}
% size rules: \large, \Large, \huge, \Huge
% more format rules: \bfseries (bold), \it (italic), \underline
% \thesection: Command to render the number of the section
\titleformat{\section}[block] {\normalfont\Large\bfseries}{\thesection.}{1em}{\Large}
% ---------------------------------------------------


% Mandatory formatting rules required by "Code of Studies and Exams" [don't change this two lines!]
% ---------------------------------------------------

% margins
\usepackage[margin=2.5cm, bindingoffset=1.25cm]{geometry}

% line spacing
\linespread{1.5}
% ---------------------------------------------------


% Useful packages
% ---------------------------------------------------

% Handle UTF8 characters
\usepackage[utf8]{inputenc}

\setlength{\emergencystretch}{10pt}

% Change numbering, e.g. to letters instead of numbers
\usepackage{enumitem}

% If you don't use \usepackage[T1]{fontenc},
%  - Words containing accented characters cannot be automatically hyphenated,
%  - You cannot properly copy-and-paste such words from the output (DVI/PS/PDF),
%  - Characters like the pipe sign, less than and greater sign give unexpected results in text.
\usepackage[T1]{fontenc}

% language specification
\usepackage[english]{babel}

% useful for equations
\usepackage{mathtools} 
% Convenience command for norm
\newcommand{\norm}[1]{\left\lVert#1\right\rVert}

\usepackage{amsmath,amssymb,amsfonts}
\DeclareMathOperator*{\argmax}{arg\,max}
\DeclareMathOperator*{\argmin}{arg\,min}
\DeclareMathOperator{\diag}{diag}

\newenvironment{tight_equations}{
    \setlength{\abovedisplayskip}{3pt}
    \setlength{\belowdisplayskip}{3pt}
}

\newcommand{\citationneeded}{~[\color{red}citation needed\color{black}]}

% greek letters in text mode (used in equation tags
\usepackage[euler]{textgreek}

% For theorems
\usepackage{amsthm}

\newtheoremstyle{def_thm}% name of the style to be used
  {20pt}% measure of space to leave above the theorem. E.g.: 3pt
  {}% measure of space to leave below the theorem. E.g.: 3pt
  {\itshape}% name of font to use in the body of the theorem
  {}% measure of space to indent
  {\bfseries}% name of head font
  {:}% punctuation between head and body
  { }% space after theorem head; " " = normal interword space
  {}% Manually specify head
\theoremstyle{def_thm}
\newtheorem*{definition}{Definition}
\newtheorem*{theorem}{Theorem}

\theoremstyle{remark}
\newtheorem*{remark}{Remark}
\newtheorem*{notation}{Notation}

\usepackage{algorithmic}
\usepackage{graphicx}
\usepackage{textcomp}
\usepackage[ruled,vlined]{algorithm2e}
%\usepackage{authblk}                     

% Format references by IEEE guidlines
\usepackage[style=ieee, backend=biber]{biblatex}

\DeclareBibliographyAlias{artwork}{misc}

% Needed for positioning the logo of the Faculty on the first page
\usepackage[export]{adjustbox}

% Display SI units
\usepackage[binary-units=true]{siunitx}

% vulgar fractions
\usepackage{textcomp}

% smaller font size for captions
\usepackage[font=footnotesize,labelfont=bf]{caption}

% Importing the hyperref package all cross-referenced elements become hyperlinked. For example, the lines in the table of contents become links to the corresponding pages in the document.
\usepackage[hidelinks, unicode, pdfusetitle]{hyperref}

% PDF bookmarks (smart pdf viewers can show this bookmarks helping the reader to navigate inside the document)
\usepackage{bookmark}

% Support for set background colors for table cells
\usepackage[table,xcdraw]{xcolor}
% colorful text
\usepackage{xcolor}

% Transform URL-s in the text clickable automatically
\usepackage{url}

% Support for long quotes
\usepackage{csquotes}

% Support for embedded source codes
\usepackage{listings}
\usepackage{sourcecodepro} % a nice font family for code fragments
\lstset{captionpos=b, numberbychapter=false, basicstyle=\ttfamily, showstringspaces=false, columns=fullflexible}

% Tell the renderer where to look for images
\graphicspath{ {images/} }

% File holding references
\addbibresource{references.bib}

% Generate Dummy Text (feel free to remove it)
\usepackage{lipsum}

% ---------------------------------------------------

% First page + empty page after it
% You should bind the the original copy of the Diploma Thesis Proposal Form (collected from the Registrar’s Office) right after this blank page!
% ---------------------------------------------------

\author{\name \\ \program}
\title{\Huge{\thesisType}\\[1cm]
    \huge{\title}}
\date{\the\year}

\newcommand{\titlePage}{
    \includegraphics[valign=m, width=50pt]{ITK_logo} \parbox[c]{0.8\textwidth}{
    Pázmány Péter Catholic University\\
    Faculty of Information Technology and Bionics}
    \vspace*{\fill}
    
    {\let\newpage\relax\maketitle}
    \vspace*{\fill}
    \begin{center}
    \bigskip
    
    \supervisors
    \end{center}
}

\newcommand{\blankPage}{
    \begingroup
        \pagestyle{empty}
        \cleardoublepage
    \endgroup
    \clearpage
}
% ---------------------------------------------------

% ------------- Dokumentum legenerálása -------------
\begin{document}

\titleformatting

% This file is responsible to collect all content and assemble the document.

% ---------------------------------------------
\def\thesisType{Master's Thesis}
\def\name{HAKKEL Tamás}
\def\program{Computer Science Engineering MSc}
\def\title{Iteratively Reweighted Algorithms for Dynamic MRI}
\def\supervisors{Supervisor: Claudio M. VERDUN \\ Faculty mentor: Dr. OLÁH András}

% Render title page
\titlePage

% Set page numbering to roman numbers for all pages before the first chapter. Change it to "\pagenumbering{Roman}" if you want upper case roman numbers
\pagenumbering{roman}

% Mandatory parts
% If you don't want these mandatory parts to be included in the table of contents, simply delete or comment out the lines starting with "\addcontentsline".
\addcontentsline{toc}{chapter}{Diploma Thesis Proposal}

% If you print your thesis two sided, the you need this blank page after the Title page
\blankPage

% The "Diploma Thesis Proposal" form takes the first two page numbers, so the "Thesis Authenticity Statement" gets the page number 3.
\setcounter{page}{3}

\addcontentsline{toc}{chapter}{Thesis Authenticity Statement}
% You have nothing to do here

\chapter*{Thesis Authenticity Statement}
I, undersigned \name, student of the Faculty of Information Technology and Bionics at the Pázmány Péter Catholic University, hereby certify that this thesis was written without any unauthorized help, solely by me and I used only the referenced sources. Every part, which is quoted exactly or in a paraphrased manner, is indicated clearly with a reference. I have not submitted this thesis anywhere else.

% Signature line
\begin{flushright}
	\vspace*{.5cm}\par\noindent\makebox[2.5in]{\hrulefill}
	\par\noindent\makebox[2.5in][c]{\name}
\end{flushright}

\clearpage
\addcontentsline{toc}{chapter}{Abstract}
\chapter*{Abstract}
% Minimum 2000 characters, but maximum two pages long substantive abstract
\paragraph{}
\lipsum[1-2] % 2 paragraphs of dummy text - replace it with your abstract
\clearpage
% Uncomment the following line if you want to include an "Acknowledgements" page after the abstract.
%\addcontentsline{toc}{chapter}{Acknowledgements}
%\chapter*{Acknowledgements}
% Optional part to give thanks to people contributing to your work (other than your supervisor and mentor) https://seleninevcilikhayati.com/accounting-dissertation-help/

\paragraph{}
\lipsum[2] % 1 paragraph of dummy text - replace it with your thanksgiving lines
\clearpage

% Render table of contents
\tableofcontents
\clearpage

% From here page numbers are arabic numbers starting again from 1
\pagenumbering{arabic}

% Include files holding the content of your thesis -- feel free to change as you like it
\chapter{Introduction}

While the fast evolution of technology profoundly changed today's medicine, unarguably the medical imaging is of the fields which profited the most of the computation power recently became available. And as X-ray radiographs revolutionized medical treatments in the beginning of the 20th century,  the appearance of computer-aided imaging techniques such as computer tomography (CT), diagnostic ultrasonography, positron emisson tomography (PET), and magnetic resonance imaging (MRI) opened a new horizon drastically increasing the resolution, allowing 3D imaging, providing reliable dynamic recordings, and enhancing images by automated post-processing. In the recent decades radiology evolved to be an interdisciplinary field involving, for instance, molecular biology, nuclear physics, applied mathematics, and computer science besides the classical medical fields such as anatomy, angiology, and cardiology.

\section{Magnetic Resonance Imaging}

\subsection{History of Medical Imaging}
The beginning of the history of medical images dates back to November of 1895 when Wilhelm Conrad Röntgen discovered X-rays. Besides the remarkable fact that he was awarded the first Nobel Prize in Physics in 1901, it also signifies the importance of his discovery that up until the 1960's it was the only medical imaging technique available. During the first 60 years of X-ray radiography, it underwent a remarkable development gradually increasing the resolution of images and decreasing the radiation dose that affected both the patients' and the doctors' health. Since the 1920s, visualization of motions (i.e. dynamic imaging) became possible by flouroscopy, although only to a limited extent~\cite{bradley_history_2008}.

While the theoretical background of later advances in medical images were present much earlier, the the introduction to clinical medicine took place only when the computers became powerful enough for image reconstruction tasks concerning both the computational power and the available memory. Therefore, the major breakthrough came only in the 1960-70s, when suddenly multiple imaging techniques were introduced. The first of these methods was the ultrasound imaging devised by Floyd Firestone in 1940 to detect internal flaws in metal castings~\cite{singh_origin_2007}, and used first for medical purposes in 1949 by John Wild~\cite{watts_john_2009}, but it was not until 1961 when David Robinson and George Kossoff developed the first commercially practical water path ultrasonic scanner~\cite{griffiths_historical_nodate}. Also, the concept of emission and transmission tomography was propsed in the late 1950s by David E. Kuhl, Luke Chapman and Roy Edwards, but computer tomography was invented in 1972 by Godfrey Hounsfield and Allan Cormack~\cite{richmond_sir_2004}, and the first PET camera was built for human studies by Edward Hoffman, Michael M. Ter-Pogossian, and Michael E. Phelps in 1973~\cite{noauthor_us_nodate}. Moreover, nuclear magnetic resonance (NMR) imaging was discovered by Felix Bloch and Edward Purcell in 1946, and Reymond Vahan Damadian proposed the first MR body scanner in 1969. Then Paul Lauterbur had the idea of applying magnetic field gradients in all three dimensions and a back-projection technique to create images in 1971, and he also published the first MRI images: water tubes, a living clam, and the thoracic cavity of a mouse in 1973 and 1974~\cite{rinck_short_2008}. Since then, these computer-aided methods continue to develop at a very fast pace, becoming essential tools for today's practitioners.

\subsection{Applications in Diagnostics}
Even though MRI is a relative new technology, it has already revolutionized medical imaging and diagnostic process as we know it. Its versatility makes it fit a wide range of use cases. Compared to other imaging technologies, MRI demonstrates important advantages in many cases. 
\begin{itemize}
    \item In contrast to X-ray, MRI doesn't use any ionizing radiation, and hence it is totally harmless to the patient. Also, MRI is much better at imaging soft tissues, in particular neural tissue,  while X-rays are rather used for diagnosing bone degeneration, dislocation, fracture, tumor and infection. Furthermore, MRI allows 3D scans. On the other hand, X-ray machines are quite inexpensive compared to MRI, and acquires the image multiple times faster.
    \item As CT scanning is also based on X-rays, it shares this downside with X-rays, doctors need to evaluate the possible benefits of the scan and decide if it outweighs the potential complications of exposure to ionizing radiations. MRI, however, elicit this problem. although at the price of a elongated imaging process. Comparing the medical problems where these technologies are used, one can conclude that CT scan is very helpful in diagnosing severe injuries of the chest, head, spine or abdomen, particularly fractures, and it is commonly used to localize tumors. An MRI, however, often performs better at diagnosing problems in the joints, soft tissues, ligaments and tendons. Doctors use it frequently to scan the spine, brain, muscles, neck, breasts, and abdomen.
    \item The strongest point of sonography is its portability, low cost, and real-time imaging speed without any harmful radiation, but this technology is rather limited to 2D imaging (although 3D imaging is possible), have trouble penetrating bone, and even in absence of bone the depth of penetration is limited depending on the frequency of imaging.
    \item PET scans are particularly useful for functional imaging; for instance, in identification of lapses in cognitive function, examination of cardiac failures, cancer screening and diagnosis, and finding an infection. The main disadvantage of this technology is nevertheless that the acquisition is relatively long (especially, if we consider also the time while patients wait for the tracer to reach the targeted organ), it uses a radioactive substance as tracer, and it cannot scan tissues not absorbing the tracer making the localization of the source of the signal challenging when no additional information is available. In practice, however, the letter limitation is solved by combining PET scanners with either CT or MRI.
\end{itemize}
To sum up, MRI is a strong competitor to other imaging technologies, but it also have weaknesses, of which the slow acquisition time tend to be the most problematic. There are many methods to speed up measurements as it will be discussed later, but the construction cost and the hardware constraints limit the applicability of these efforts. The problem of slowness is even more apparent in case of dynamic images as motion of organs (e.g. heart or lung) can drastically degrade the image quality. To overcome that issue, software solutions get more and more attention, especially since the appearance of compressed sensing based reconstruction methods.

\section{Compressed Sensing}
\subsection{Core Concept}
\subsection{Application Areas}
\subsection{Suitability to MRI}

\section{Julia Language}
\subsection{Objectives of The Language}
\subsection{Suitability to Our Task}

\section{Objective}

% ----------------------------------------------------
\section{Outline}
The summary of the chapters of the thesis work:

\paragraph{Chapter 2} This chapter describes something and here I summarize it in a couple sentences.

\paragraph{Chapter 3} This chapter describes something and here I summarize it in a couple sentences.

\paragraph{Chapter 4} This chapter describes something and here I summarize it in a couple sentences.

\paragraph{Chapter 5} This chapter describes something and here I summarize it in a couple sentences.

\clearpage % You need \clearpage at the end of every chapter to force images included in this chapter to be rendered in somewhere else
\chapter{Theoretical Background of MRI}

\section{History of Medical Imaging}
The beginning of the history of medical images dates back to November of 1895, when Wilhelm Conrad Röntgen discovered X-rays. The significance of his discovery is well demonstrated by the fact that up until the 1960's X-rays were the only non-invasive way to look into the body, and hence he was awarded the first Nobel Prize in Physics in 1901 for \emph{``in recognition of the extraordinary services he has rendered by the discovery of the remarkable rays subsequently named after him''}~\cite{nobelprize_1901}. During the first 60 years of X-ray radiography, it underwent a remarkable development gradually increasing the resolution of images and decreasing the radiation dose that threatened both the patients' and the doctors' health. Since the 1920s, visualization of motions (i.e. dynamic imaging) became possible by fluoroscopy, although only to a limited extent~\cite{bradley_history_2008}.

Although the theoretical background of later advances in medical images were present much earlier, e.g., by the seminal work of Johann Radon, for the introduction of modern imaging technologies to clinical medicine are delayed to the time when the computers became powerful enough for image reconstruction tasks concerning both the computational speed and the available memory. Therefore, the major breakthrough came only in the 1960-70s, and then suddenly multiple imaging techniques were introduced shortly after each other.

The first of these methods was the ultrasound imaging. It was devised by Floyd Firestone in 1940 to detect internal flaws in metal castings~\cite{singh_origin_2007}, and it was proposed for medical purposes first in 1949 by John Wild~\cite{watts_john_2009}, but it was not until 1961 when David Robinson and George Kossoff developed the first commercially practical water path ultrasonic scanner~\cite{griffiths_historical_nodate}.

Similarly, the concept of emission and transmission tomography was proposed in the late 1950s by David E. Kuhl, Luke Chapman and Roy Edwards, but computer tomography was invented only in 1972 by Godfrey Hounsfield and Allan Cormack~\cite{richmond_sir_2004}, and the first PET camera was built for human studies by Edward Hoffman, Michael M. Ter-Pogossian, and Michael E. Phelps in 1973~\cite{noauthor_us_nodate}.

Finally, the youngest imaging technology, the MRI, became available for diagnostics by the end of the 1970s, even though the nuclear magnetic resonance (NMR) spectroscopy that has the same mechanism was discovered by Felix Bloch and Edward Purcell already in 1946. Then Reymond Vahan Damadian proposed the first MR body scanner in 1969, and soon Paul Lauterbur had the idea of applying magnetic field gradients in all three dimensions and a back-projection technique to create images in 1971, and he also published the first MRI images: water tubes, a living clam, and the thoracic cavity of a mouse in 1973 and 1974~\cite{rinck_short_2008}.

Since their appearance, these computer-aided methods continue to develop at a very fast pace, becoming essential tools for today's practitioners despite the fact that they are relative new technologies.

\section{Physics of MRI}

This section attempts to dive into the physics of MRI giving a quick overview of the theory of nuclear magnetic resonance following~\cite{nishimura_principles_1996, kurzhunov_novel_2017, pooley_fundamental_2005}.

\subsection{Components of MRI machines}
The theory of measurements based on nuclear magnetic resonance has its root in quantum physics: The nuclear magnetic moment and the angular momentum of protons in the atomic nuclei maintained by the spin of these particles are to be indirectly measured, and these observables depend (besides many other factors) on the tissue where the proton is located. More specifically, the MRI
machines are tuned to focus on the nucleus of protons that consist of only one proton. The core components of MRI machines are the following:
\begin{enumerate}
    \item Superconductive coils immersed in liquid helium are the largest and most expensive part of the machine. They are responsible for producing a almost perfectly homogeneous and static magnetic field. The role of the liquid helium is to keep the wires at superconducting temperature, so that massive amounts of electricity can be run through the coils creating super-strong fields up to \SI{21.1}{\tesla}~\cite{schepkin_vivo_2012}. Although stronger magnetic field allows better resolution, the construction costs of such machines and the effect of the strong magnetic field on human tissues limit the strength of available MRI scanners for routine clinical from \SI{0.2}{\tesla} to \SI{3.0}{\tesla}, and up to \SI{11.7}{\tesla} in research machines for human imaging~\cite{ladd_pros_2018}.
    \item Inside of this super-strong electromagnet, the so called gradient coils are located that alter the field along all three dimensions creating spatially varying magnetic field (hence the name: gradient coils) in order of \SI{}{\milli\tesla}, so that signals coming from different location within the coils are possible to be separated. They are also used to provide contrast for diffusion and flow imaging.
    \item Within the Radio Frequency (RF) coils are located that emit and measure time varying electromagnetic signals on order of tens of \SI{}{\micro\tesla}.
\end{enumerate}
The reason behind this elaborate design (depicted on fig.~\ref{fig:mri_schematic}) is the need of creating a measurement setup suitable to give a very fine control over the the direction of the magnetic moment of protons of hydrogen atoms within the measured object (which, in our case, is the human body that contains a large amount of hydrogen mostly in the form of water, but also bounded within other molecules).

\begin{figure}
    \centering
    \includegraphics[width=.5\linewidth]{images/mri-scanner.jpg}
    \caption{\textbf{Schematic illustration} of construction of a cylindrical MRI scanner. Source:~\cite{coyne_mri_2020}.}
    \label{fig:mri_schematic}
\end{figure}

\subsection{Macroscopic Magnetization}
The purpose of the superconductive coils is to align the magnetic moment of protons with the direction of the magnetic field. This direction (also corresponding to the head-to-foot direction) is usually referred to as longitudinal direction or z direction, and the plane perpendicular to this direction is called the transverse plane or the x-y plane. This alignment of the magnetic moment of the protons leads to two configurations: protons with their magnetic moment pointing to the same direction as the static magnetic field, and other protons having their magnetic moment with opposite direction. Without the static field, the randomly oriented spins cancel out each other, as they also do in the aligned case, when the number of protons oriented to the two directions are equal. But in real systems, a slight excess of the protons aligned with the static magnetic field always produces a net magnetization with the same direction as the external magnetic field (see fig.\ref{fig:net_magnetization}).

The ratio of the number of protons in these two groups are described by the Fermi-Dirac statistics. In strong and static magnetic field at room temperature, the Fermi-Dirac distribution reduces to Boltzmann distribution resulting the following formula:
\[N_+ = N \cdot \frac{e^{E_+ / (k_B T)}}{e^{E_+ / (k_B T)} + e^{E_- / (k_B T)}} \text{ and } N_- = N \cdot \frac{e^{E_- / (k_B T)}}{e^{E_+ / (k_B T)} + e^{E_- / (k_B T)}},\]
where $N$ is the total number of protons, $N_+$ and $N_-$ are the numbers of protons pointing to the same and opposite direction as the static magnetic field, $E_+$ and $E_-$ are their respective energy levels, $k_B$ is the Boltzmann-constant, and $T$ is the temperature. In this case neighboring energy levels are equidistant with the difference in the secondary spin quantum number of $\Delta m = \pm 1$ and the energy difference of $\nabla E = \gamma \hbar B_0$, where $\gamma$ is an empirical constant called gyromagnetic ratio (equals to $42.575 \cdot 2\pi$\SI{}{\mega\hertz/\tesla} in case of protons), $\hbar$ is the reduced Planck constant, and $B_0$ is the static external magnetic field. The ratio of Boltzmann distributions for two states a spin \textonehalf nucleus is known as the Boltzmann factor:
\[f(E) = e^{-\frac{\gamma \hbar B_0}{k_B T}}.\]
Using this factor, the ratio of unpaired protons (these protons give the net magnetization) divided by the number of all protons is given by
\[\frac{N_+ - N_-}{N_+ + N_-} = \frac{\gamma \hbar B_0}{2 k_B T}.\]
This ratio at room temperature in a static field with a couple teslas is a tiny number (in the order of \num{1e-6} multiplied by $B_0$), so that explains why do MRI machines need such a strong electromagnets. (Note that this ratio also can be increased by increasing the temperature, but it is not feasible for human imaging.)

\begin{figure}[tb]
    \centering
    \begin{minipage}{.52\textwidth}
        \centering
        \includegraphics[width=0.8\linewidth]{images/net_magnetization.pdf}
        \caption{\textbf{Effect of strong external magnetic field ($B_0$):} Spins of protons get aligned with the field in either parallel or anti-paralellel direction producing a net magnetization ($M$) parallel with the external field.}
        \label{fig:net_magnetization}
    \end{minipage}%
    \hspace{0.03\textwidth}
    \begin{minipage}{0.44\textwidth}
        \centering
        \includegraphics[width=0.4\linewidth]{images/precession.pdf}
        \caption{\textbf{Precession of protons.} As a result of RF excitation pulse, the magnetic momentum of protons deviates from the longitudinal direction and starts to precess due to its angular moment.}
        \label{fig:precession}
    \end{minipage}
\end{figure}

\subsection{Precession}
In equilibrium when all protons are aligned with the external magnetic field, the longitudinal component of net magnetization is maximal and the component in the transverse plane is zero. However, with the aid of an electromagnetic excitation in the transverse plane emitted by the RF coils, it is possible to rotate the vector of net magnetization into the transverse plane. The key factor in this process is to tune the frequency of the excitation to match the so called precessional frequency of the protons given by the Larmor equation:
\[\omega = \gamma B_0.\]
The name \textit{precessional frequency} comes from the phenomenon that the magnetic moment of protons start to precess around the longitudinal axis (which, again, is the direction of static external magnetic field) due to its intrinsic angular momentum. When the frequency of the excitation matches the precessional frequency of the proton (which happens to be in the radio frequency range, hence the name of RF coils), then resonance occurs and the angle of net magnetization gets tilted (illustrated by fig.~\ref{fig:precession}), otherwise the electromagnetic field has little to no effect of the net magnetization. An RF excitation of a duration $\tau$ causes rotation of the magnetization by an angle $\theta$, which is called the flip angle, defined by
\[\theta = \gamma \int_0^\tau B_1(t) dt = \gamma \tau B_1,\]
where $B_1$ is the magnetization of RF excitation, and it is assumed to be constant over time window of excitation with length $\tau$.

As a result of the precession, the net magnetic flux changes in the RF coils (these coils used for both emitting and receiving RF signals) inducing an electromotive force $U_{ind}$ that can be calculated by Faraday's law of induction:
\[U_{ind} = -\frac{d\Phi}{dt}.\]
Projecting the precessing movement (with the Larmor frequency $\omega$) of the net magnetization to transversal plan, we get a sinusoidal change in flux that results in the following formula:
\[U_{ind} \sim sin(\theta)\, \omega\, cos(\omega t) = sin(\theta)\, \gamma\, B_0\, cos(\gamma\, B_0\ t).\]
While this formula is not an exact model that fits the current measured in the RF coils, but it captures three important aspects of the resulted electric signal:
\begin{itemize}
    \item It is a sinusoidal signal with a frequency depending only on a constant specific to protons and the external magnetic field.
    \item The amplitude of that signal depends on the flip angle induced by an RF excitation.
    \item And it is also dependent on the external magnetic field (yet another reason why MRI machines need very strong electromagnets).
\end{itemize}

\subsection{Relaxation}
For a more accurate model, one should consider that as the protons emit RF signal due to their precessing magnetic moment, they lose the energy of the excitation and they slowly return to the low energy state; i.e., to the state where the magnetic moment of protons are aligned along the longitudinal axis, and where the net magnetization points to the same direction as the external field. Assuming that the excitation resulted in a perpendicular flip angle, the longitudinal component of magnetization is characterized by the exponential formula
\[M_z = M_0 (1 - e^{-t/T_1}),\]
where $M_0$ is the amplitude of magnetization in the equilibrium and is often called Boltzmann magnetization, and $T_1$ time constant is a property of the protons dependent on the tissue where they are located. The name of this process is $T_1$ relaxation.

\begin{figure}[tb]
    \centering
    \includegraphics[width=0.8\linewidth]{images/T1_relaxation.png}
    \caption{\textbf{$T_1$ relaxation after a \SI{90}{\degree} RF excitation.} By the end of the pulse, the magnetization rotated from the z-direction to the x-y plane; therefore, the longitudinal component is reduced to zero, and gradually it returns to the equilibrium. Source:~\cite{ridgway_cardiovascular_2010}.}
    \label{fig:T1_relaxation}
\end{figure}

Furthermore, the net magnetization is also affected by another relaxation process called $T_2$ relaxation. The phenomenon causing this relaxation is called \textit{dephasing}, and the name comes from the fact that when excitation is applied to protons in the equilibrium, they will precess in the same phase, but soon they lose this synchronization. This desynchronization is due to the slight inhomogeneity of the static external field caused by four factors: spin-spin interactions (quantum mechanical interactions with the nearby protons), magnetic field inhomogeneities (hardware limitations), magnetic susceptibility (slight magnetization of molecules within the measured part of the body), and chemical shift effects (shielding effect of the electron cloud of molecules incorporating the hydrogen atoms). The slightly different $B_0$ value makes the Larmor frequency different, and that results in the desynchronization of phase. The outcome of this process is that the transversal component of the net magnetization decays exponentially to zero. The speed of decay is characterized by the $T_2^*$ time constant:
\[M_{xy} = M_1\,e^{-t/T_2^*},\]
where $M_1$ is the initial amplitude of net magnetization in the beginning of the $T_2$ relaxation process. For illustration of this process, see fig.~\ref{fig:T1_relaxation}. The resultant decaying signal is known as the Free Induction Decay (FID). Using, however, the later described \textit{spin-echo} acquisition protocol, the last three inhomogeneity-causing factors can be cancelled out leading to a slightly different time constant denoted by $T_2$.

\begin{figure}[tb]
    \centering
    \includegraphics[width=0.8\linewidth]{images/T2_relaxation.png}
    \caption{\textbf{$T_2$ relaxation process.} Right after the RF pulse, thre precession of all protons are in the same phase, but they quickly desynchronize due to magnetic field inhomogeneities. The plot in the second row depicts the strength of current induced in the RF coils that corresponds to the projection of the x-y component ($M_{xy}$) of the net magnetization along the direction perpendicular to the surface of the coil. And as a result of the $T_2$ relaxation, the amplitude of this sinusoidal curve exponentially decays. The resultant decaying signal is known as the Free Induction Decay (FID). Source:~\cite{ridgway_cardiovascular_2010}.}
    \label{fig:T2_relaxation}
\end{figure}

\section{Concepts of MR Imaging}

In that section, the most important concepts of MR imaging are summarized, clarifying the vocabulary used in the later chapters, based on~\cite{nishimura_principles_1996, pooley_fundamental_2005}.

\subsection{MRI Sequences}
Since the advent of MRI, numerous methods were developed and are used in today's medicine. And while they are all measure somehow the $T_1$ and $T_2$ constants at different location, the produced image is quite different, making them fit different use cases. These methods are called MRI sequences and they mostly differ in the a particular setting of RF pulses and the gradients in the static magnetic field, resulting in a particular image appearance. The most commonly used group of MRI sequences is the \textit{spin echo}~\cite{hahn_spin_1950}. In accordance with the two types of relaxation, sequences in that group have two main parameters: $TR$ (Time of Repetition) and $TE$ (Time of Echo). These parameters have a crucial role timing the recording of current in the RF coils when the difference between the amplitude of the RF signal emitted by the excited protons is maximal because this difference makes it possible later to distinguish different tissues.

The $TR$ parameter is connected to the $T_1$ value, as it determines the time between two excitation pulses. Having a larger $TR$ value allows protons to get better aligned with the external magnetic field before the next excitation, which results in a higher initial value for $M_{xy}$, leading to a stronger current in the detector coils, but it also makes the entire measurement longer. On the other hand, $TE$ determines the delay between the peak of the RF pulse and the peak of the echo. That echo is a temporary rephasing of spins caused by a second, \SI{180}{\degree} RF pulse emitted at $t = TE/2$. That pulse inverts spins, and therefore it makes spins with slower Larmor frequency, which lagged behind the faster ones previously, be ahead of the others in phase. At the time when faster precessing protons catch up, the transversal magnetization exhibits an echo peak (see~\ref{fig:spin_echo}). As stated earlier, an important advantage of spin echo technique is that three factors of magnetic inhomogeneity is cancelled out by the inversion as these factors are constant over time, while spin-spin interactions are random interactions between protons that cause random local changes in the magnetic fields experienced by the protons.

\begin{figure}[tb]
    \centering
    \includegraphics[width=0.8\linewidth]{images/spin_echo.png}
    \caption{\textbf{Process of spin echo sequence.} The echo is a temporary rephasing of spins caused by a second, \SI{180}{\degree} RF pulse emitted at $t = TE/2$. That pulse inverts spins, and therefore it makes spins with slower Larmor frequency, which lagged behind the faster ones previously, be ahead of the others in phase. At the time when faster precessing protons catch up, the transversal magnetization exhibits an echo peak. Source:~\cite{ridgway_cardiovascular_2010}.}
    \label{fig:spin_echo}
\end{figure}

Before moving forward, an important thing to note that the $T_1$ relaxation is much slower than the $T_2$ relaxation ($T_1$ relaxation takes hundreds of milliseconds up to a few seconds while $T_2$ rarely exceeds \SI{200}{\milli\second}. As a result, the acquisition time is mostly dominated by waiting for the $T_1$ relaxation, and therefore short $TR$ values are favorable when fast imaging is needed. Also, the different time-scale of $T_1$ and $T_2$ relaxation opens a range of possibilities to make acquisition process faster or more effective.

Based on the choice of $TR$ and $TE$ values, we can talk about three types of spin echo sequences: $T_1$ weighted sequence has intermediate $TR$ value in the magnitude of $T_1$ producing maximal T1 weighting (at this point, the difference caused by different $T_1$ value between the amplitude of signals coming from different tissues are maximal) and short $TE$ value magnitudes smaller than $T_2$ producing minimal T2 weighting (there is not enough time to have significant difference between decay curves with different $T_2$). To the contrary, T2-weighted images have a long $TE$ (maximizing the difference in $T_2$ relaxation) and long $TR$ (reducing the weight of $T_1$ relaxation). And the third type, called proton density (PD) weighting, uses short $TE$ and long $TE$, so that the pixel intensities on the resulted image will reflect only the density of protons (that also differs between tissues), and the $T_1$ and $T_2$ values have little effect on it.

\begin{figure}[tb]
    \centering
    \begin{minipage}[t]{.31\textwidth}
        \centering
        \includegraphics[width=.7\linewidth]{images/T1_weighted.png}
        \caption{\textbf{$T_1$ weighted sequence.} Choosing the $TR$ value to be relatively short, and $TE$ to be relatively short, the difference due to the variation of $T_1$ value over tissues would dominate over differences caused by different $T_2$ value. Source: Adapted from~\cite{ridgway_cardiovascular_2010}.}
        \label{fig:T1_weighted}
    \end{minipage}%
    \hspace{0.02\textwidth}
    \begin{minipage}[t]{.28\textwidth}
        \centering
        \includegraphics[width=\linewidth]{images/T2_weighted.png}
        \caption{\textbf{$T_2$ weighted sequence.} Choosing both $TR$ and $TE$ values to be relatively long, the difference due to the variation of $T_2$ would dominate over differences caused by different $T_1$ value. Source: Adapted from~\cite{ridgway_cardiovascular_2010}.}
        \label{fig:T2_weighted}
    \end{minipage}%
    \hspace{0.02\textwidth}
    \begin{minipage}[t]{0.34\textwidth}
        \centering
        \includegraphics[width=.85\linewidth]{images/PD_weighted.png}
        \caption{\textbf{Proton density (PD) weighted sequence.} Choosing the $TR$ value to be relatively long, and $TE$ to be relatively short, the difference due to the variation of both $T_1$ and $T_2$ values are minimized, and hence mostly the density of protons would determine signal strength. Source: Adapted from~\cite{ridgway_cardiovascular_2010}.}
        \label{fig:PD_weighted}
    \end{minipage}
\end{figure}

Two common variant of spin echo are multiecho spin-echo, and turbo spin-echo. The multiecho spin-echo pulse sequence utilizes multiple \SI{180}{\degree} RF pulses to induce multiple echo peaks each with a different $TE$, forming multiple images of the same object with different weighting ranging from PD-weighting to $T_2$-weighting. This method exploits the fact that $T_1$ is much larger than $T_1$, thus multiple echos can be performed without drastically changing the acquisition time. Similarly, turbo spin-echo sequences consist of multiple echo-generating \SI{180}{\degree} pulses, but in this case only one image is formed speeding up the imaging by gathering information about multiple positions in each cycle.

The other large group of sequences is the gradient echo sequence~\cite{winkler_characteristics_1988}. That type of sequences differs from echo-spin that the flip angle of initial RF pulse is less than \SI{90}{\degree} (e.g., \SI{20}{\degree} or \SI{30}{\degree}) and there is no \SI{180}{\degree} secondary pulse, instead it induces an echo by the spacial gradients explained later. Hence this method is able to perform the measurement much faster as $T_1$ relaxation reaches near-equilibrium state much earlier.

Beyond these sequence types, several other commonly used variants exist, which will not be discussed here, such as the inversion recovery sequences~\cite{dwyer_short-ti_1988, fleckenstein_fast_1991, ashgriz_flair_1991, bedell_implementation_1998}, diffusion-weighted sequences~\cite{moseley_diffusion-weighted_1990, bammer_basic_2003}, perfusion weighted sequences~\cite{rosen_perfusion_1990, detre_perfusion_1992, barbier_methodology_2001}, BOLD-contrast images for functional MRI (fMRI)~\cite{ogawa_brain_1990, kwong_dynamic_1992}.

\subsection{Spatial Encoding}
The core concept that allowed the extension of NRM (which is based on the same principles described above) to MRI, is the spatial encoding. This technique, proposed by Paul Lauterbur, allows for the localization of RF emitting protons or, more precisely, the localization of an \textit{ensemble} of RF emitting protons within a small volume called voxels (note that the size and the shape of these voxels are defined by the configuration of MRI machine for the given acquisition). Specifically, the problem with static magnetic field is that, even though the net magnetization varies over the measured object based on the type of the tissue, the RF pulse excites the entire volume of the measured object, and therefore the induced signal of each voxel sum up making it impossible to separate them based on their position. In contrast, generating a secondary magnetic field with a gradient along a specific direction makes the Larmor frequency dependent on the position along that direction. That dependence can be exploited multiple ways allowing exact localization along all three dimensions. A possible (and quite common) way to do this is the following:
\begin{enumerate}
    \item Producing a gradient along the z-direction during the RF excitation permits the selection of a slice perpendicular to the z-axis by tuning the RF pulse to the frequency specific to the given slice because this pulse excites then only protons in the selected slice. In reality, however, not a single frequency used but rather a band of frequencies whose width matches the bandwidth of resonance frequencies of spins in the slice of interest. This approximately rectangular band of excitation frequencies is realized in time domain by a pulse of a shape similar to the sinc function.
    \item Application of another gradient along the y-direction between the RF excitation and the readout of induced current causes a gradual de-synchronization of phase along the y-axis because protons experiencing higher external field will have a higher Larmor frequency as well. Therefore, the position along y-direction becomes \textit{phase-encoded}.
    \item During the time window of readout, another gradient along the x-direction can be used to separate signal sources along that dimension by the difference in their frequency. This type of spacial encoding is called \textit{frequency encoding}.
\end{enumerate}
These steps are visually explained on fig.~\ref{fig:slice_selection} and~\ref{fig:phase_and_freq_encoding}.

\begin{figure}[tb]
    \centering
    \begin{minipage}{.43\textwidth}
        \centering
        \includegraphics[width=\linewidth]{images/slice_selection.png}
        \caption{\textbf{Mechanism of slice selection.} Producing a gradient along the z-direction during the RF excitation permits the selection of a slice perpendicular to the z-axis by tuning the RF pulse to cover the frequency range specific to the given slice with given thickness because this pulse excites then only protons in this slice. Source:~\cite{ridgway_cardiovascular_2010}.}
        \label{fig:slice_selection}
    \end{minipage}%
    \hspace{0.03\textwidth}
    \begin{minipage}{0.53\textwidth}
        \centering
        \includegraphics[width=\linewidth]{images/phase_and_freq_encoding.png}
        \caption{\textbf{Process of phase and frequency encoding.} First, a gradient along the y-direction between the RF excitation and the readout causes a gradual de-synchronization of phase along the y-axis because protons experiencing higher external field will have a higher Larmor frequency. Second, a gradient along the x-direction can be used to separate signal sources along that dimension by the difference in their frequency. Source:~\cite{ridgway_cardiovascular_2010}.}
        \label{fig:phase_and_freq_encoding}
    \end{minipage}
\end{figure}

Decoding the spatial information becomes feasible then using the famous Bloch equation:
\[\frac{d\textbf{M}}{dt} = \textbf{M} \times \boldsymbol{\gamma} \textbf{B} - \frac{M_x\textbf{i} + M_y\textbf{j}}{T_2} - \frac{(M_z - M_0)\textbf{k}}{T_1}.\]
Knowing that the transversal component of the net magnetization (that is, the measured component) is non-zero only a selected slice and is dependent on the position within the slice assuming the excitation strategy above, a convenient formulation is to use a complex-valued function to denote the amplitude of the signal generated at position $(x,y)$ by $m(x,y) = m_x(x,y) + m_y(x,y)$, where $m_x(x,y)$ and $m_y(x,y)$ is the amplitude realized in a coil perpendicular to x-axis and y-axis, respectively. As a result of the phase-encoding, the phase of a spin at position $y$ along y-axis is given by $\omega_0 t_y + \gamma G_y t_y y,$ where $\omega_0$ is the base Larmor frequency of the selected slice, $t_y$ denotes the width of time window of phase-encoding, and $G_y$ corresponds to the slope of the linear gradient applied along the y-direction. Because $m(x,y)$ is a complex-valued function representing the magnetization in both the x and y-direction, the distribution $m(x,y)$ gets weighted by a complex exponential corresponding to the spatial frequency $\gamma/2\pi)G_y t_y$: $m(x,y)exp(-i\gamma G_y t_y y$. 

Similarly, the shift in the frequency of precession during the readout characterized by $\gamma G_x x$ adds another weighting term to $m(x,y)$. Solving the Bloch equation for this configuration results in the following formula:
\[s(t) = \int_x \int_y m(x,y) e^{-i\gamma G_x x t} e^{-i\gamma G_y y t_y} dx dy,\]
where $s$ is the amplitude of signal to be measured, $t_y$ (length of time window for phase-encoding) is a fixed number and $t$ is a running variable. Having a look at the formula, one can immediately see that it is the 2-dimensional Fourier transform of $m(x,y)$ corresponding to the coordinate $((\gamma/2\pi)G_x t, (\gamma/2\pi)G_y t_y)$ in the Fourier space. Due to the discrete nature of this the acquisition process, the MRI machines can evaluate the Fourier space in discrete positions, that is, the collected data is the \textit{discrete Fourier transform} (DFT) of the object. Consequently, the space of measurements is usually referred to as k-space. This connection between the measured signal and Fourier transformation is beneficial in many aspects, as it will be apparent in the next sections.

\subsection{Sampling Trajectories}
The effect of the spatial encoding schema discussed above is that the k-space points can only be measured sequentially. Combined with the necessity of waiting for the $T_1$ relaxation (that tends to be in the timescale of  \SIrange{250}{1500}{\milli\second} at \SI{3}{T}, and somewhat shorter for \SI{1.5}{T}~\cite{gold_musculoskeletal_2004,bojorquez_what_2017,stanisz_t1_nodate}), it leads to the most important limitation of MR imaging: the acquisition is quite slow. While it is inconvenient for the patient to stay inside the narrow scanner bore (especially for claustrophobic patients), the more important issue is that the longer is the acquisition the more motion artifacts are introduced to the image. The type of such involuntary motions include bulk motion (e.g., coughing, change body position), respiratory motion, cardiac motion, and motion of other organs like blood vessels or parts of the gastrointestinal tract. Although some of them can be reduced by a large extent, for example, by asking patient to pay attention to remain still or to hold breath for a \SIrange{10}{20}{\second} long measurement, others are out of control. Reducing the k-space points, however, lead to various aliasing effects according to Nyquist-Shannon sampling theorem that might blur clinically relevant features or introduce misleading artifacts. Therefore, radiologists always seek to find an optimal compromise between the number of measured k-space points and the amount of motion artifacts introduced.

One important aspect of dealing with that problem is determining which points are to be measured. Due to hardware limitations and because too rapidly changing electromagnetic field would overheat the measured tissues, arbitrary positioning in k-space is not feasible, and thus the order of measurement points are also a key aspects of the process. These constrains introduces the necessity of well-defined geometries describing the location and the order of measured k-space coordinates, called sampling trajectories.

The traditional way of acquiring K-space data is through Cartesian trajectories. The concept of these trajectories is that equally spaced grid-points are selected in the slice or volume to be measured, and these points are evaluated systematically; for instance, row-by-row, zig-zag, or spirally. The main benefit of this method that a simple and fast reconstruction is possible via fast Fourier transform (FFT). Nevertheless, this method fails to exploit a very important feature of natural images in general: the energy tends to be concentrated in the center of the Fourier space; that is to say, the coefficients of lower frequencies have usually large magnitudes, and points further from the center have mostly a near-zero value. The significance of this observation is perfectly demonstrated by the success of the modern image and video compression algorithms making use of this uneven distribution, for instance, JPEG, MP3, and standards. Hence, multiple non-equidistant sampling trajectories are also used in practice. Such trajectories are radial~\cite{rasche_continuous_1995}, spiral~\cite{blum_fast_1987}, concentric rings~\cite{wu_mri_2008}, and 3D cones trajectories~\cite{gurney_design_2006}, just to name a few. A couple 3-dimensional trajectories are depicted in fig.~\ref{fig:trajectories}.

\begin{figure}[tb]
    \centering
    \includegraphics[width=0.8\linewidth]{images/trajectories.png}
    \caption{\textbf{A few example for 3D trajectories.} Source:~\cite{noauthor_forschungszentrum_nodate}}
    \label{fig:trajectories}
\end{figure}

\subsection{Accelerated MRI}\label{section:accelerated}
For a deeper understanding of accelerating methods, one needs to get familiarized with the concept of field of view and its connection to the bandwidth of RF excitation pulse. The term \textit{field of view} (FOV) refers to the area or volume over which an MR image is acquired, and often also to its size as well. Due to the discrete nature of digital systems, the FOV is also discretized representing it with a grid of equidistant points whose values in grid points can then be displayed as pixels on computer screens. The two main parameters of FOV are therefore its size and the pixel width. The size of FOV is proportional to the bandwidth of RF excitation and inverse proportional to the strength of the frequency encoding gradient. Thus, it makes increasing the RF bandwidth is beneficial for the image quality (although is comes with increased production costs as well), and it limits the strength of the frequency encoding gradient, even though stronger gradient would be advantageous to better separate tissues by their precessional frequencies. In contrast, the pixel size is determined by the size of range over which k-spaces samples are collected. In case of equidistant sampling in k-space over a square shaped FOV, the connection between these parameters can be expressed by simple equations:
\[\Delta k = 1 / FOV \text{ and } \Delta w = 1 / k_{FOV},\]
where $\Delta k$ is the distance between k-space points, $FOV$ refers to the size of the FOV, $\Delta w$ is the pixel width, and $k_{FOV}$ is defined as the range between the highest positive and largest negative spacial frequencies in k-space (i.e., $k_{FOV} = k_{max}^+ - k_{max}^- = 2 \cdot k_{max}$, if $k_{max}^+ = |k_{max}^-|$). For a visual example, please refer to fig.~\ref{fig:cartesian}.

\begin{figure}[tb]
    \centering
    \includegraphics[width=0.6\textwidth]{images/cartesian.png}
    \caption{\textbf{Connection between FOV and k-space} in case of Cartesian sampling: The size of FOV is inversely proportional to the distance of k-space points, and the pixel size is determined by the size of range over which k-spaces samples are collected. Source:~\cite{noauthor_kspace_nodate}.}
    \label{fig:cartesian}
\end{figure}

Using these concepts, one can formulate many methods to accelerate acquisition. For instance, it is possible to evaluate only half of the k-space exploiting the underlying symmetry in the Fourier transform. This method is successful in maintaining a high resolution, albeit it come with the cost of reduced signal-to-noise ratio (SNR) because of the lack of the noise-cancelling effect of two-sided measurements. Another popular method is to avoid acquiring the periphery of k-space by limiting the range of the phase-encoding frequencies (which, in fact, are connected to $T_1$ relaxation that contributes the most to the acqisition time). The advantage of this method is that it produces a decent speed-up keeping the SNR relatively high, but undersampling the high frequencies have a blurring effect, and therefore the resolution goes down. Finally, it is also possible to reduce the size of the FOV leading to a rectangular shaped image (also by reducing the phase-encoding step). The speed-up here, however, also comes with a cost: the amount of noise unfortunately remains the same as in the case of the wider FOV, but now it is distributed over a smaller area, and consequently the SNR also goes down.

A similar, but more effective method is proposed to accelerate imaging using multiple independent RF coils are built around the measured object. That method is commonly referred to as parallel imaging (PI), and idea behind is that the placement and the sensitivity characteristics of the coils are known, therefore the amplitude of the measured signal can be used to assist the localization of the signal source. This additional information makes less phase-encoding steps sufficient for acquisition, and thus allows a potentially several-fold reduction in imaging time. For an illustrative example, see~\ref{fig:parallel_imaging}. While PI also have downsides, namely the increased production cost of PI-capable machines, the unavoidable reduction in SNR (because each coil has its own, independent noise that sums up at reconstruction), and the introduction of PI-specific artifacts that comes from the inaccurate estimation of the coils sensitivities over the FOV, and the uneven distribution of noise related to coil geometry, the effect of these drawbacks can be reduces by the increase of number of coils (albeit this further increases the production cost).

\begin{figure}[tb]
    \centering
    \includegraphics[width=0.6\linewidth]{images/parallel_imaging.jpg}
    \caption{\textbf{Illustrative example for parallel imaging technique.} Multiple independent RF coils are built around the measured object, and as the detection sensitivities of the coils are known, the amplitude of the measured signal can be used to assist the localization of the signal source. Source:~\cite{hamilton_recent_2017}}
    \label{fig:parallel_imaging}
\end{figure}

\subsection{Bottom Line}
Since the advent of MR imaging, an explosion of the amount of MRI concepts and technologies is witnessed by the scientific community during the last 3-4 decades. Although the quality of the images has undergone drastic improvement, placing MRI to the focus of today's diagnostic, the speed of evolution based on hardware-related innovations seems to slow down as the technology approaches its physical limits (e.g., the strength of the static magnetic field cannot be increased infinitely and the number of coils in PI is limited, for example, by the space available inside the machine). And while the acquisition time is significantly improved over time, but MRI is still considered to be a slow imaging technique, making dynamic imaging a challenging task. Hence, there is an increasing interest towards software solutions which can push further down the number of k-space points required to produce images with a quality sufficient for successful diagnosis.

\clearpage
\chapter{Mathematical Foundation}

\section{Basics of Compressed Sensing}
As it was shortly mentioned in chapter~\ref{chapter:introduction}, a mathematical framework called compressed sensing revolutionized MR image acquisition process allowing reconstruction from much fewer k-space values \textit{under certain conditions} as it would be necessary according to the Nyquist criterion.

To realize this promise, first and foremost, the signal to be recovered must be sparse in some transform domain. Fortunately, natural images are intrinsically sparse in the Fourier domain, MR images are being no exception to that, and as MRI scanners operates on Fourier transform, the sparsity condition is always satisfied by the very nature of the imaging process. The other conditions, however, are less intuitive, hence in this section we attempt to give a quick overview of the most important definitions and theorems needed for basic understanding, based on the book~\cite{foucart_mathematical_2013}, and on the lectures of the course titled \textit{Compressive Sampling} at Technical University of Munich by Alihan Kaplan.

\subsection{Elementary Definitions}

Although the reader might be to be familiar with the most of these definitions, for the sake of completeness and clarity of notation used in this work, we present here a list of definitions of elementary constructs, restricting ourselves to mere formulations with short remarks omitting further explanation.

\begin{tight_equations}

\begin{definition}[norm]
A non-negative function $\norm{\cdot}: X \rightarrow [0, \infty)$ is called a norm, if 
\begin{enumerate}[label=\alph*)]
    \item $\norm{\mathbf{x}} = 0$ if and only if $\mathbf{x} = \mathbf{0}$,
    \item $\norm{\lambda \mathbf{x}} = \Vert \lambda \Vert \norm{\mathbf{x}}$ for all scalars $\lambda$ and all vectors $\mathbf{x} \in X$, and
    \item $\norm{\mathbf{x} + \mathbf{y}} \le \norm{\mathbf{x}} + \norm{\mathbf{y}}$ for all vectors $\mathbf{x, y} \in X$.
\end{enumerate}
\end{definition}

\begin{remark}
$X$ denotes a vector space on which the norm is defined. In MRI setting, however, $\mathbb{C}^N$ is the default vector space for computations, and therefore, we also define the following constructs in this space.
\end{remark}

\begin{definition}[$\ell_p$-norms for vectors]
The $\ell_p$-norm on $\mathbb{C}^N$ is defined for $1 \le p < \infty$ as
\begin{equation}\label{eq:p-norm}
\norm{\mathbf{x}}_p = \left(\sum_{j=1}^n |x_j|^p\right)^{\frac{1}{p}},
\end{equation}
and for $p = \infty$ as
\[\norm{\mathbf{x}}_\infty = \max_{j \in [n]} | x_j |.\]
For $0 < p < 1$, (\ref{eq:p-norm}) defines a quasinorm, which means that from the definition of the norm a) and b) holds, but c) is replaced by the weaker quasitriangle inequality
\[\norm{\mathbf{x} + \mathbf{y}}_p \le C\left(\norm{\mathbf{x}}_p + \norm{\mathbf{y}}_p\right)\]
with $C = 2^{\frac{1}{2}-1}$.
\end{definition}

\begin{notation}
$[n]$ denotes the set of integers form $0$ to $n-1$.
\end{notation}

\begin{remark}
Using the schema above, it is impossible to have a proper norm for $p = 0$; nonetheless, is very common to define $\ell_0$-norm as the number of non-zero coordinates:
\[\norm{\mathbf{x}}_0 = \left| \left\{x_i \ne 0 : i \in [n]\right\}\right| \text{ where } \mathbf{x} \in \mathbb{C}^N.\]
Following this convention, $\norm{\cdot}_0$ always refers to that definition in the definitions and theorems below.
\end{remark}

\begin{definition}[sparsity]
We call a vector $s$-sparse, if at most $s$ of its entries are non-zero; i.e. $\norm{\mathbf{x}}_0 \le s$.
\end{definition}

\begin{notation}
By $\Sigma_s^N$ we denote the set of all s-sparse vectors in $\mathbb{C}^N$; that is,
\[\Sigma_s^N = \left\{\mathbf{x} \in \mathbb{C}^N : \norm{\mathbf{x}}_0 \le s \right\}.\]
\end{notation}

%\begin{notation}
%The solution set $\left\{\mathbf{x} \in \mathbb{C}^N : \mathbf{Ax} = \mathbf{y}\right\}$ is denoted by $L_\mathbf{A}(\mathbf{y})$ for a given $\mathbf{A} \in \mathbb{C}^{m \times N}$.
%\end{notation}

%\begin{definition}[unique sparse solution]
%A vector $\mathbf{x}_* \in \mathbb{C}^N$ is the unique $s$-sparse solution of equation $\mathbf{Ax} = \mathbf{y}$ with $\mathbf{A} \in \mathbb{C}^{m \times N}$ and $\mathbf{y} \in \mathbb{C}^m$, if $L_\mathbf{A}(\mathbf{y}) \cap \Sigma_s^N = \{\mathbf{x}_*\}$, in other words, it is the only $s$-sparse vector satisfying the  linear system defined by $\mathbf{A}$ and $\mathbf{y}$.
%\end{definition}

\begin{definition}[kernel/null space]
The kernel/null space of a matrix $\mathbf{A} \in \mathbb{C}^{m \times N}$ is defined as
\[ker(\mathbf{A}) = \left\{\mathbf{x} \in \mathbb{C}^N : \mathbf{Ax} = \mathbf{0}\right\}.\]
\end{definition}

\begin{theorem}[SVD]
For $\mathbf{A} \in \mathbb{C}^{m \times N}$, there exist unitary matrices $\mathbf{U} \in \mathbb{C}^{m \times m}$, $\mathbf{V} \in \mathbb{C}^{N \times N}$, and uniquely defined non-negative numbers $\sigma_1 \ge \sigma_2 \ge \ldots \ge \sigma_{min\{m,N\}} \ge 0$ called singular values of $\mathbf{A}$, such that 
\[\mathbf{A} = \mathbf{U \Sigma V^*} \text{ where } \mathbf{\Sigma} = diag[\sigma_1, \sigma_2, \ldots, \sigma_{min\{m,N\}}] \in \mathbb{R}^{m \times N}.\]
The process of obtaining these matrices is called Singular Value Decomposition (SVD).
\end{theorem}

\end{tight_equations}

\subsection{Formulation of the Problem}
In engineering settings, especially in signal processing context, engineers and scientist usually try first to model physical systems by a linear model because that way they describe the problem by a set of linear expressions, and then they can express it as a matrix-vector multiplication $\mathbf{Ax} = \mathbf{y}$, where vector $\mathbf{x}$ is the input of the system, vector $\mathbf{y}$ is the output (measured data), and $\mathbf{A}$ characterizes the measurement (thus it is often referred to as \textit{measurement matrix}). A very common task then is to recover the input consisting of $N$ variables from the measurement data, and generally the number of measurements $m$ must obey $m > N$, otherwise the linear system is underdetermined, and hence there exist infinitely many solutions. In case of MRI setting, this statement corresponds to already mentioned Nyquist criterion that requires the sampling frequency in k-space to be twice as the highest frequency in the image space (vid. $k_{FOV} = 2 \cdot k_{max}$ in section~\ref{section:accelerated}).

And that is the point when compressed sensing comes into play claiming that given a \textit{proper} measurement matrix $\mathbf{A}$, the problem

\begin{equation}
    \tag{P\textsubscript{0}}\label{eq:P_0}
    \min_{\mathbf{z} \in \mathbb{C}^N} \norm{\mathbf{z}}_0 \text{ subject to } \mathbf{Az} = \mathbf{y} = \mathbf{Ax}
\end{equation}
have a unique $s$-sparse solution with $s \ll N$. This optimization problem is often referred to as (\ref{eq:P_0}). The number of necessary measurement, however, still a difficult question. There are theoretical results stating that $m = 2s$ is the lower bound for a perfect recovery~\citationneeded and that a stable recovery (later explained) occurs with high probability with $m \ge C s log(N / m)$ for random measurement matrices~\citationneeded, but in practice, reconstruction algorithms struggle to reach these theoretical limits (albeit, the achieved $m$ is still drastically improved compared to the Nyquist sampled case).

The main reason why the optimal bounds are usually not reached is that the measurement matrices of real life systems have often have less favorable properties as random matrices and, more importantly, (\ref{eq:P_0}) is a NP-hard problem~\citationneeded; thus, only relaxations can be solved. The most commonly used relaxations are the so called $\ell_1$ minimization or Basis Pursuit (BP)~\citationneeded defined as
\begin{equation}
    \tag{P\textsubscript{1}}\label{eq:P_1}
    \min_{\mathbf{z} \in \mathbb{C}^N} \norm{\mathbf{z}}_1 \text{ subject to } \mathbf{Az} = \mathbf{y},
\end{equation}
the Basis Pursuit DeNoising (BPDN)~\citationneeded formulated as

\begin{equation}
    \tag{P\textsubscript{1}, \texteta}\label{eq:P_1_noisy}
    \min_{\mathbf{z} \in \mathbb{C}^N} \norm{\mathbf{z}}_1 \text{ subject to } \norm{\mathbf{Az} - \mathbf{y}}_2 \le \eta,
\end{equation}
and the LASSO (Least Absolute Shrinkage and Selection Operator)~\citationneeded problem expressed by
\[\min_{\mathbf{z} \in \mathbb{C}^N} \norm{\mathbf{Az} - \mathbf{y}}_2 \text{ subject to } \norm{\mathbf{z}}_1 \le s.\]
By the aid of a Lagrangian multiplier $\lambda$, the latter two can be transformed to the same unconstrained minimization problem
\[\min_{\mathbf{z} \in \mathbb{C}^N} \norm{\mathbf{Az} - \mathbf{y}}_2 + \lambda \norm{\mathbf{z}}_1.\]

\subsection{Conditions and Guarantees}
While $\ell_1$-relaxation (often referred to as (\ref{eq:P_1}) problem) might be appealing as it can be solved efficiently in polynomial time, it needs a more careful approach to guarantee that the minimum of the $\ell_1$ problem is also a solution of (\ref{eq:P_0}). The $\ell_2$-relaxation, for example, always has a unique solution, but this solution is not necessarily sparse. In contrast, $\ell_1$ problem has a solution which is both unique and $s$-sparse given that the matrix $\mathbf{A}$ fulfills the so called \textit{null space property} of order $s$, introduced by Cohen, Dahmen and DeVore in~\cite{cohen_compressed_2009}.

\begin{definition}[NSP]
A matrix $\mathbf{A} \in \mathbb{C}^{m \times N}$ is said to satisfy the null space property (NSP) of order $s$, if for any set $S \subset [N]$ with $|S| = s$
\[\norm{\mathbf{v}_S}_1 < \norm{\mathbf{v}_{S^C}}_1 : \forall \mathbf{v} \in ker(\mathbf{A})  \setminus \{\mathbf{0}\}.\]
\end{definition}

\begin{notation}
For a vector $\mathbf{v} \in \mathbb{C}^N$ and a set $S \subset [N]$, we denote by $\mathbf{v}_S$ either the vector in $\mathbb{C}^{|S|}$ which is the restriction of $\mathbf{v}$ to the indices in $S$, or the vector in $\mathbb{C}^N$ which coincides with $\mathbf{v}$ on the indices in $S$ and is zero elsewhere. Similarly, $\mathbf{v}_{S^C}$ means the same with the complement of $S$.
\end{notation}

\begin{theorem}
Given a matrix $\mathbf{A} \in \mathbb{C}^{m \times N}$, every $s$-sparse vector $\mathbf{x} \in \Sigma_s^N \subset \mathbb{C}^N$ is the unique solution of (\ref{eq:P_1}) with $\mathbf{y} = \mathbf{Ax}$ if and only if $\mathbf{A}$ satisfies the NSP of order $s$.
\end{theorem}

\begin{remark}
This theorem shows that for every $\mathbf{y} = \mathbf{Ax}$ with $s$-sparse $\mathbf{x}$, the $\ell_1$-minimization (\ref{eq:P_1}) actually solves the $\ell_0$-minimization (\ref{eq:P_0}) when the NSP of order $s$ holds.
\end{remark}

Although NSP is a formidable construct that allows relatively easy and straightforward proofs, in realistic settings, signals are rarely sparse, but rather \textit{almost} $s$-sparse vectors, meaning that the most part of the energy of the signal is concentrated in $s$ coefficients with the the largest values. To involve these cases in the compressed sensing framework, an extension of NSP is used, called \textit{stable NSP}. Proving the stable NSP property, one can then approximate the almost $s$-sparse signal with the \textit{best $s$-term approximation} having a tight bound on the error term.

\begin{definition}[$\ell_p$-error of best $s$-term approximation]
For $p > 0$, the $\ell_p$-error of best $s$-term approximation to a vector $\mathbf{x} \in \mathbb{C}^N$ is defined by
\[\sigma_s(\mathbf{x})_p = \inf\left\{\norm{\mathbf{x} - \mathbf{z}}_p : \mathbf{z} \in \Sigma_s^N\right\}.\]
\end{definition}

\begin{remark}
The  infimum is always achieved by an $\mathbf{z} \in \Sigma_s^N$ whose non-zero entries equal the $s$ largest absolute entries of $\mathbf{x}$.
\end{remark}

\begin{definition}[stable NSP]
A matrix $\mathbf{A} \in \mathbb{C}^{m \times N}$ is said to satisfy the stable NSP with constant $0 < \rho < 1$ of order $s$, if for any set $S \subset [N]$ with $|S| = s$
\[\norm{\mathbf{v}_S}_1 \le \rho \norm{\mathbf{v}_{S^C}}_1 : \forall \mathbf{v} \in ker(\mathbf{A}).\]
\end{definition}

\begin{theorem}
The matrix $\mathbf{A} \in \mathbb{C}^{m \times N}$ satisfies the stable NSP with constant $0 < \rho < 1$ of order $s$ if and only if for any set $S \subset [N]$ with $|S| = s$
\begin{equation}\label{eq:stable_nsp}
    \norm{\mathbf{z} - \mathbf{x}}_1 \le \frac{1 + \rho}{1 - \rho} \left(\norm{\mathbf{z}}_1 - \norm{\mathbf{x}}_1 + 2\norm{\mathbf{x}_{S^C}}\right)
\end{equation}
holds for any set $S \subset [N]$ with $|S| = s$ and for all vectors $\mathbf{x,z} \in \mathbb{C}^N$ with $\mathbf{Az} = \mathbf{Ax}$.
\end{theorem}

\begin{remark}
The estimation (\ref{eq:stable_nsp}) can be upper-bounded by means of the $\ell_1$-error of best $s$-term approximation $\sigma_s(\mathbf{x})_1$ as
\[\norm{\mathbf{z} - \mathbf{x}}_1 \le \frac{1 + \rho}{1 - \rho} \left(\norm{\mathbf{z}}_1 - \norm{\mathbf{x}}_1 + 2\norm{\mathbf{x}_{S^C}}\right) \le 2 \cdot \frac{1 + \rho}{1 - \rho}\sigma_s(\mathbf{x})_1.\]
\end{remark}

The significance of this theorem is that it implies that having the stable NSP fulfilled, the unique $s$-sparse solution $\mathbf{z}$ of (\ref{eq:P_1}) with $\mathbf{y} = \mathbf{Ax}$ approximates the vector $\mathbf{x}$ with a bounded $\ell_1$-error. Therefore, by a good approximation of the sparsity $s$ of the signal and an upper bound $\epsilon$ of $\sigma_s(\mathbf{x})_1$ (both can be inferred by the statistics of the signal), one can construct or select a measurement matrix $\mathbf{A}$ such that the solution of \textit{any} algorithm solving (\ref{eq:P_1}) is an approximation of the input signal with the maximal error of
\[\frac{1 + \rho}{1 - \rho} \cdot 2\epsilon,\]
where $\rho$ depends only on the measurement matrix and $\epsilon$ is a small number because only little energy is stored in the entries contributing the the $\ell_p$-error of best $s$-term approximation.
Practically that means that by the careful design of the measurement, arbitrary precision recovery is possible (having Nyquist sampling as a special case for guaranteed perfect recovery), and that in case of almost $s$-sparse signals, the number of measurements can be drastically reduced without significant amount of error introduced.

Finally, this construct can be further extended  to handle noisy measurement: proving the \textit{robust NSP} for the measurement matrix, the same guaranties apply to the $\ell_1$-error as in case of stable recovery, extended by an extra term characterizing the noise on the measurement. As a result, arbitrary precision is allowed for arbitrary optimization algorithm capable of solving (\ref{eq:P_1_noisy}) given a measurement matrix satisfying the \textit{robust NSP}.

\begin{definition}[robust NSP]
A matrix $\mathbf{A} \in \mathbb{C}^{m \times N}$ is said to satisfy the robust NSP with constants $0 < \rho < 1$ and $0 < \tau$ of order $s$, if for any set $S \subset [N]$ with $|S| = s$
\[\norm{\mathbf{v}_S}_1 \le \rho \norm{\mathbf{v}_{S^C}}_1 + \tau \norm{\mathbf{Av}}_2 : \forall \mathbf{v} \in \mathbb{C}^N.\]
\end{definition}

\begin{remark}
Note that $\mathbf{v}$ is not required to be in $ker(\mathbf{A})$.  In fact, if $\mathbf{v} \in ker(\mathbf{A})$, then $\mathbf{Av} = \mathbf{0}$, and we obtain the definition of stable NSP. Therefore, the robust NSP implies stable NSP.
\end{remark}

\begin{theorem}
The matrix $\mathbf{A} \in \mathbb{C}^{m \times N}$ satisfies the robust NSP with constants $0 < \rho < 1$ and $0 < \tau$ of order $s$ if and only if
\begin{equation}\label{eq:robust_nsp}
    \norm{\mathbf{z} - \mathbf{x}}_1 \le \frac{1 + \rho}{1 - \rho} \left(\norm{\mathbf{z}}_1 - \norm{\mathbf{x}}_1 + 2\norm{\mathbf{x}_{S^C}}\right) + \frac{2\tau}{1-\rho}\norm{\mathbf{A(z-x)}}_2
\end{equation}
holds for any set $S \subset [N]$ with $|S| = s$ and for all vectors $\mathbf{x,z} \in \mathbb{C}^N$ with $\mathbf{Az} = \mathbf{Ax}$.
\end{theorem}

\begin{remark}
Supposing that a matrix $\mathbf{A} \in \mathbb{C}^{m \times N}$ satisfies the robust NSP of order $s$, (\ref{eq:robust_nsp}) implies that for any $\mathbf{x} \in \mathbb{C}^N$, a solution $\mathbf{z}$ of (\ref{eq:P_1_noisy}) with $\norm{\mathbf{Ax - y}} \le \eta$ approximates $\mathbf{x}$ with $\ell_1$-error \footnote{Number $4$ in the numerator of the error term at (\ref{eq:error_bound}) is not a typo as one would expect, but a counterintuitive result of some basic arithmetics leading to this expression.}
\begin{equation}\label{eq:error_bound}
    \norm{\mathbf{z} - \mathbf{x}}_1 \le 2 \cdot \frac{1 + \rho}{1 - \rho}\sigma_s(\mathbf{x})_1 + \frac{4\tau}{1-\rho}\eta.
\end{equation}
\end{remark}


\iffalse

\section{First Order Gradient Methods}

My slideshow about first order methods: \url{https://docs.google.com/presentation/d/1tIZSSfzzHUgo9tlKlmhSUCCWD3ynzyZwnWtpBUe--6g/edit}

\subsection{Gradient Descent}
From Cauchy to recent explosion of optimization algorithm.

The nonlinear conjugate gradient method is mainly based on the same idea as gradient descent (GD) method: That method can be imagined as a hiker trying to get down from a mountain and that hiker always follows the steepest direction; i.e., he or she is moving along the negative gradient in each time step. On the other hand, while the basic GD is easy to understand, fairly efficient, and widely used method to solve unconstrained optimization problems, it has also its drawbacks compared to other variants of the method. The three main problems are 1) the optimal step size, 2) curved "valleys", and 3) flat areas on the cost function surface. Fig.~\ref{fig:grad_desc_problem} shows an example where the inappropriate step size and a "curved valley" produce zigzagging motion, and slow convergence as a result, which is slowed even further when the flat area in the bottom of that valley is reached.

\begin{figure}
    \centering
    \includegraphics[width=0.3\linewidth]{images/project with Wiem/Banana-SteepDesc.png}
    \caption{Example of the effect of inappropriate step size, "curved valley" and flat area: slow convergence. Image from Wikipedia.org.}
    \label{fig:grad_desc_problem}
\end{figure}

Why would second order methods speed up convergence and why aren't we able to apply them to our problem.

\subsection{Conjugate Gradient}
How do they work, and when can conjugate gradient outperform gradient descent

\subsection{Conjugate Gradient Method}
To overcome the drawbacks of the basic GD method, one can use the conjugate gradient method (CG) that is an improved version of the basic GD method: in that case, the direction of movement must be conjugate to the previous directions. Two non-zero vectors $\mathbf{u}$ and $\mathbf{v}$ are conjugate (with respect to some $\mathbf{A}$ matrix), if $$\mathbf{u}^T\mathbf{Av} = \mathbf{0}.$$
Let us consider the following linear system as the subject of optimization:
$$\mathbf{Ax} = \mathbf{b},$$
where $\mathbf{A} $is symmetric, positive-definite and real matrix, and $\mathbf{b}$ is also known.

\subsubsection{As a direct method} Since $\mathbf{A}$ is symmetric and positive-definite, it defines an inner product:
$$\langle \mathbf{u}, \mathbf{v} \rangle_\mathbf{A} := \mathbf{u}^T\mathbf{Av}$$
Using that inner product, it is possible to find $n$ pairwise conjugate vectors: $\mathcal{P} = \{\mathbf{p}_1,...\mathbf{p}_n\}$. Then $\mathcal{P}$ forms a basis in $\mathbb{R}^n$, so the solution ($\mathbf{x_*}$) of optimization problem can be represented in terms of that basis:
\begin{equation} \label{eq:direct_method}
    \mathbf{x_*} = \sum_{i=1}^{n} \alpha_i \mathbf{p}_i.
\end{equation}
Left multiplying both sides with $\mathbf{p}_k^T \mathbf{A}$ we get:
$$\mathbf{p}_k^T\mathbf{Ax_*} = \sum_{i=1}^{n} \alpha_i \mathbf{p}_k^T \mathbf{A} \mathbf{p}_i = \sum_{i=1}^{n} \alpha_i \langle \mathbf{p}_k, \mathbf{p}_i \rangle_\mathbf{A}.$$
As we know that $\mathbf{Ax_*} = \mathbf{b}$ and that $\langle \mathbf{p}_k, \mathbf{p}_i \rangle_\mathbf{A} = 0 : \forall i \ne k $  because $\mathbf{p}_i$ vectors are mutually conjugate (i.e. orthogonal with respect to the inner product defined by matrix $\mathbf{A}$):
$$\mathbf{p}_k^T\mathbf{b} = \alpha_k \langle \mathbf{p}_k^T, \mathbf{p}_k \rangle_\mathbf{A}.$$
That way we can calculate all $\alpha_i$ coefficients,:
$$\alpha_i = \frac{\mathbf{p}_i^T\mathbf{b}}{\langle \mathbf{p}_i, \mathbf{i}_k \rangle_\mathbf{A}},$$
and using these coefficients $\mathbf{x_*}$ can be calculated directly by equation \ref{eq:direct_method}.

\subsubsection{As an iterative method} One weakness of direct method that for high dimensional vectors, we have to calculate high number of coefficients. On the other hand, it is not necessary to calculate all of them as a good approximation of $\mathbf{x_* }$ can be obtained using only a few well chosen $\mathbf{p_i}$ vectors. To achieve that we need to define a cost function:
$$f(\mathbf{x}) = \frac{1}{2}\mathbf{x}^T \mathbf{Ax} - \mathbf{x}^T \mathbf{b}.$$
The existence of a unique minimizer is evident as its second derivative is given by a symmetric positive-definite matrix:
$$\nabla^2 f(\mathbf{x}) = \mathbf{A}.$$
Also, we can calculate the first derivative easily:
$$\nabla f(\mathbf{x}) = \mathbf{Ax} - \mathbf{b}.$$
After choosing an arbitrary starting point $\mathbf{x}_0$, we can start the iteration by calculating the so called \textit{residual}, which is apparently equal to the negative gradient:
$$\mathbf{r}_{k+1} = \mathbf{b} - \mathbf{Ax}.$$
Then we have to make sure to get a direction that is conjugate to all previous directions by applying an operation similar to the Gram-Schmidt orthonormalising:
$$\mathbf{p}_k = \mathbf{r}_k - \sum_{i<k} \frac{\langle \mathbf{p}_i, \mathbf{r}_k \rangle_\mathbf{A}}{\langle \mathbf{p}_i, \mathbf{p}_i \rangle_\mathbf{A}} \mathbf{p}_i.$$
Following this direction, the next optimal location is given by
$$\mathbf{x}_{k+1} = \mathbf{x}_k + \alpha_k \mathbf{p}_k,$$
where $\alpha_k$ can be derived by substituting the previous formula for $\mathbf{x}_{k+1}$ to the the cost function and minimizing it with respect to $\alpha_k$ :
$$\nabla f(\mathbf{x}_{k+1}) = \nabla f(\mathbf{x}_k + \alpha_k \mathbf{p}_k)  \overset{!}{=} 0 \Rightarrow ... \Rightarrow \alpha_k = \frac{\mathbf{p}_k^T \mathbf{r}_k}{\langle \mathbf{p}_k, \mathbf{p}_k \rangle_\mathbf{A}}$$
Fig.~\ref{fig:CG_vs_GD} depicts the difference between CG and GD methods by showing an example when CG reaches the optimum in two steps, while GD needs approx. 6 steps.

\begin{figure}
    \centering
    \includegraphics[width=0.3\linewidth]{images/project with Wiem/Conjugate_gradient_illustration.png}
    \caption{Visualization of the advantage of the CG method compared to GD: CG requires significantly fewer steps. The green line shows the path of the GD method, red shows the path of the CG method. Image from Wikipedia.org.}
    \label{fig:CG_vs_GD}
\end{figure}

\subsection{Nonlinear Conjugate Gradient Method}
Our problem, however, is nonlinear, so the CG method described above needs some further adjustment. Therefore, the nonlinear conjugate gradient (NCG) method is used to minimize the cost function $f(\mathbf{x})$, and thus reconstruct the image:
$$\mathbf{x}_{k+1} = \mathbf{x}_k + \alpha_k \mathbf{d}_k$$
\begin{align*}
    \mathbf{d}_k &=
    \begin{cases}
    -\mathbf{g}_k & k = 1 \\
    -\mathbf{g}_k + \beta_k \mathbf{d}_{k-1} & k > 1
    \end{cases},
\end{align*}
where $\mathbf{g}_k$ denotes the gradient at the $k$th step.

While that method appears to be even simpler than the CG method, unfortunately, there is no optimal way to calculate $\alpha_k$ and $\beta_k$ parameters, so multiple methods are proposed. The two main conditions of a good method are the following:
\begin{itemize}
    \item Descent property:
    \begin{itemize}
        \item Strict: Cost function must strictly decrease in each step ($\mathbf{g}_k \mathbf{d}_k < 0$)
        \item Sufficient: We can allow a certain amount of increase in cost function ($\mathbf{g}_k \mathbf{d}_k < -c ||\mathbf{g}_k||^2$)
    \end{itemize}
    \item Global convergence
    \item Fast convergence
\end{itemize}
While it is practically impossible to fully satisfy all these three conditions at the same time, but these can be used to compare the different methods.

1) To get the $\alpha_k$ parameter, the most commonly used method is line search:
$$\alpha_k = \argmin_{\alpha > 0} f(\mathbf{x}_k + \alpha_k \mathbf{d}_k).$$
Since exact line search is usually expensive and impractical, the strong Wolfe line search is often considered in the implementation of nonlinear conjugate gradient methods. It aims to find a step size satisfying the strong Wolfe conditions:
$$f(\mathbf{x}_k + \alpha_k \mathbf{d}_k) - f(\mathbf{x}_k) \leq \rho \alpha_k \mathbf{g}_k \mathbf{d}_k$$
$$|g(\mathbf{x}_k + \alpha_k \mathbf{d}_k)^T \mathbf{d}_k| \leq - \sigma \mathbf{g}_k^T \mathbf{d}_k.$$
One method to find an $\alpha$ satisfying the Wolfe conditions is backtracking line search, where the "query" point is initially moved to a certain distance from the current location along the selected direction, and then that distance is decreased in each step until the local minimum is reached.
Fig.~\ref{fig:backtracking} visualizes that method in case of a simple quadratic function.

\begin{figure}
    \centering
    \includegraphics[width=0.3\linewidth]{images/project with Wiem/backtracking.png}
    \caption{Example of backtracking line search.}
    \label{fig:backtracking}
\end{figure}

2) Similarly, to obtain the $\beta_k$, there are multiple choices. The oldest method is the Fletcher-Reeves (FR) method:
$$\beta_k^{FR} = \frac{||\mathbf{g}_k||^2}{||\mathbf{g}_{k-1}||^2}.$$
The advantage of that method that global convergence is proved for $\sigma < \frac{1}{2}$ in the Wolfe condition using inexact line search~\cite{Al-Baali}. However, the convergence is relatively slow in many cases because it may fall into some circle of tiny steps, so that method is significantly outperformed by the Polak-Ribi\`{e}re-Polyak (PRP) method:
$$\beta_k^{FR} = \frac{\mathbf{g}_k^T (\mathbf{g}_k - \mathbf{g}_{k-1}}{||\mathbf{g}_{k-1}||^2}.$$
In spite of the improved speed, that method also has a serious problem: Global convergence can only be proved for strictly quadratic functions. There exist some sophisticated line search method that ensures global convergence for all nonlinear functions, but these methods are also computationally more intensive~\cite{grippo}.
To overcome that problem, one can improve the convergence compared to FR method speed (but reduce compared to PRP) while keeping the global convergence property for Wolfe line search by combining the two methods: $\beta_k^{GN} = \max\{-\beta_k^{FR}, \min\{\beta_k^{PRP},\beta_k^{FR}\}\}.$
Although there are multiple newer methods that are more efficient, these methods fall out of the scope of that study.

3) To sum up, the algorithm used in \cite{sparse} is the following ($TolGrad$, $maxIter$, and $\mu$ variables are parameters to control the precision):
\begin{algorithm}[H]
 // Initialization:\\
 $k \leftarrow 0$; $m \leftarrow 0$; $\mathbf{g}_0 \leftarrow \nabla f(\mathbf{x}_0);  \mathbf{d}_0 \leftarrow - \mathbf{g}_0$\;
 -------------------------------------------------------------------- \\
 // Iterations:\\
 \While{$||\mathbf{g}_k\|_2 < TolGrad$ and $k > maxIter$}{
  // Backtracking line search\\
   ~~~~~~~(inner loop condition: 1st Wolfe condition)\\
  $\alpha_k \leftarrow 1$\\
  \While{$f(\mathbf{x}_k + t  \mathbf{d}_k) - f(\mathbf{x}_k) > \rho \alpha_k \cdot \mathbf{g}_k^*  \mathbf{d}_k$}{
    $\alpha_k \leftarrow \mu \cdot \alpha_k$\\
  }
 ---------------------------------------------------------------- \\
  // Changing position along the selected direction using the calculated $\alpha_k$\\
  $\mathbf{x}_{k+1} \leftarrow \mathbf{x}_{k} + \alpha_k \mathbf{d}_{k}$\\
 ---------------------------------------------------------------- \\
  // Calculate next direction\\
  $\mathbf{g}_{k+1} \leftarrow \nabla f(\mathbf{x}_{k+1})$\\
  $\gamma \leftarrow \frac{||\mathbf{g}_{k+1}||_2^2}{||\mathbf{g}_{k}||_2^2}$ // Fletcher-Reeves method\\
  $ \mathbf{d}_{k+1} \leftarrow - \mathbf{g}_{k+1} + \gamma \mathbf{x}_{k}$\\
  $k \leftarrow k + 1$\\
 }
\end{algorithm}


\subsection{Proximal Methods}
Proximal operator, Iterative Shrinkage/Soft Thresholding Algorithm, Forward-Backward Splitting with linear line search, Wolfe conditions

To solve the optimization problem that leads to image reconstruction, a simple and widely applied method is the gradient descent and its derivatives (e.g. conjugate gradient); however, they cannot be applied simply in our case because the cost function
$$f(\mathbf{x}) = \frac{1}{2} \sum_{\ell = 1}^L \sigma_\ell^{-2} \|F_\Omega \mathbf{S}_\ell \mathbf{x} - \mathbf{y}_\ell \|_2^2 + \lambda \|\mathbf{\Phi x} \|_1$$
does not have a derivative as the $\ell^1$ norm cannot be differentiated.
Finding a both universal and efficient method to solve constrained optimization problems, where either cost function or constraint does not have a gradient, is still unsolved, but we can solve that issue by restricting the class of problems to those that has the following general Lagrangian form:
$$\mathbf{\hat{x}} = \argmin_{x \in \mathcal{H}} \{E(\mathbf{x}) + R(\mathbf{x})\},$$
where $E(\mathbf{x})$ (empirical error) is continuously differentiable convex function with $\beta$-Lipschitz continuous gradient ($\|f(\mathbf{x}) - f(\mathbf{y})\| \leq \beta \|x - y\| : \forall \mathbf{x}, \mathbf{y} \in \mathbb{C}^N$), and $R(\mathbf{x})$ (regularization term) is a non-smooth, continuous function. For that class of problems, we can define the proximity operator instead of the gradient to find a local minimum in the proximity of the current location:
$$prox_f(\mathbf{v}) := \argmin_x \frac{1}{2} \|\mathbf{x} - \mathbf{v} \|_2^2 + f(\mathbf{x}).$$

\subsection{Forward-backward splitting}
One possible implementation of proximity gradient method is forward-backward splitting (FB), which consists of two consecutive steps: First, the "forward step" that optimizes $E(x)$ with gradient ($\mathbf{w}_{k+1} = \mathbf{x}_k - t_k \nabla E(\mathbf{x}_k)$), then the "backward step" optimizes $R(\mathbf{x})$ by proximity operator ($\mathbf{x}_{k+1} = prox_R(\mathbf{w}_{k+1})$). That two steps can be merged into one formula:
$$\mathbf{x}_{k+1} = prox_R(\mathbf{x}_k - t_k \nabla E(\mathbf{x}_k)).$$

\subsection{ISTA: Iterative Shrinkage-Thresholding Algorithm}
The most difficult part of FB splitting is the calculation of proximity operator in general. However, the problem can be tremendously simplified by restricting regularization term to $\ell^1$-norm, also known as LASSO (least absolute shrinkage and selection operator) regularization: $R(\mathbf{x}) = \lambda ||\mathbf{\Phi x} ||_1$. That way the proximity operator can be simplified to soft-thresholding function~\cite{combettes_wajs_2005}:
$$soft(x,c) = sign(x) \cdot max(|x| - c, 0).$$
Fig.~\ref{fig:soft-thres} shows an example for soft-thresholding functions. Because of that name, the algorithm is also commonly referred as iterative \textit{soft}-thresholding algorithm. In that special case of FB splitting, it is even possible to determine an upper bound for convergence, which is $\mathcal{O}(\frac{1}{\epsilon})$ in that case~\cite{FISTA}.

\subsection{Relaxed Forward-Backward}
While ISTA is a well-known and widely used optimization method, its convergence speed can be vastly improved by taking the "momentum" into account. As the name implies, there is an analogy from physics that explains well how that method works: Let's imagine a ball rolling down from a hill. That ball, obviously, always try to follow the gradient, but as it rolls down in one direction, it also gains speed, thus momentum and kinetic energy. If the direction gradient changes frequently, than the ball also changes directions often, and it cannot gain speed. But if there is a straight slope, then the ball gathers kinetic energy, and it takes more time to change the direction even if the direction of the gradient changes.

That phenomenon can be expressed by the momentum term, which is basically the numerical derivative of the position, and we adjust the current position by that term:
$$x_k = prox_{ \gamma || \cdot ||_1}(z_k - \frac{1}{L} \nabla E(z_k)),$$
$$z_{k+1} = x_k + \mu (x_k - x_{k-1}),$$
where $\mu$ is the weight that balances the effect of momentum and gradient. However, determining the $\mu$ value is not trivial, and the optimal value depends on the problem. Such a measurement that searches for the optimal value is presented in~\cite{peyre_2011}, and fig.~\ref{fig:mu} shows the result.

%\begin{figure}
%    \centering
%    \includegraphics[width=0.5\linewidth]{relaxed-FB.png}
%    \caption{Speed of convergence for different $\mu$ values. %Image from~\cite{peyre_2011}.}
 %   \label{fig:mu}
%\end{figure}

\subsection{FISTA: Fast Iterative Shrinkage-Thresh. Algorithm}
Whereas introducing momentum rule leads to significant gain in convergence speed, the upper bound of convergence is still $\mathcal{O}(\frac{1}{\epsilon})$, and determining an optimal $\mu$ is a problem, as well. However, Beck et al.~\cite{FISTA} presented a method that can determine $\mu$ easily in such a way that the rate of convergence is $\mathcal{O}(\frac{1}{\epsilon^2})$, which is the theoretical limit defined by Nesterov~\cite{nesterov_1983} for optimization methods:
$$x_k = prox_{ \gamma || \cdot ||_1}(z_k - \frac{1}{\beta} \nabla E(z_k)),$$
$$\tau_k = \frac{1 + \sqrt{1 + 4(\tau_{k-1})^2}}{2},$$
$$\mu_k = \frac{\tau_{k-1} - 1}{\tau_k},$$
$$z_{k+1} = x_k + \mu_k (x_k - x_{k-1}),$$
where $\beta$ can easily calculated by power iteration method (eigenvalue decomp.) because of NFFT.

The reason why that method is significantly faster than the previous method is that $\mu$ value changes through the optimization, so the effect of momentum is small in the beginning of the optimization process (as the gradient is large enough to provide good convergence speed), but later the importance of momentum is gradually increasing because the surface of cost function is flat around the solution in most cases (see fig.~\ref{fig:mu_FISTA}). That method provides even faster convergence than relaxed FB, and also solve the problem of finding optimal $\mu$ value. Fig.~\ref{fig:ISTA_vs_FISTA} shows a comparison of converge speed in case of a simple optimization problem presented in~\cite{peyre_2011}.

%\begin{figure}
 %   \centering
 %   \includegraphics[width=0.5\linewidth]{mu.png}
 %   \caption{Change of $\mu$ value during optimization steps.}
 %   \label{fig:mu_FISTA}
%\end{figure}

%\begin{figure}
 %   \centering
 %   \includegraphics[width=0.5\linewidth]{ISTA_vs_FISTA.png}
 %   \caption{Comparism of convergence speed of FB, relaxed FB, %and FISTA methods. Image from~\cite{peyre_2011}.}
 %   \label{fig:ISTA_vs_FISTA}
%\end{figure}

\subsection{FOGM: Proximal Optimized Gradient Method}
Another method to increase convergence speed is proposed in~\cite{hendrickx_2018}, which was further improved in~\cite{gueddari_2018}. The main features of that method is that it gives a changing weight ($\gamma_k$) to the proximity operator, uses a more advanced form momentum rule, and it increases $\tau$ value in the last two steps increasing the effect of momentum at the same time that helps to avoid the slowdown of the convergence in the flat area around the solution. Although these modifications do not change the theoretical lower bound for convergence speed ($\mathcal{O}(\frac{1}{\epsilon^2})$), they make the algorithm having an about two-times faster worst-case convergence speed compared to FISTA~\cite{kim, taylor}. The steps of that algorithm are the following:

\begin{algorithm}[H]
 $k \leftarrow 0$; $\tau_0 \leftarrow 1$; $\mathbf{y}_0, \mathbf{z}_0 \leftarrow$ arbitrary value\;
 \While{$k \leq K - 1$}{
  \eIf{$k < K - 1$}{
    $\tau_k \leftarrow \frac{1 + \sqrt{1 + 4(\tau_{k-1})^2}}{2}$\\
  } {
    $\tau_k \leftarrow \frac{1 + \sqrt{1 + 8(\tau_{k-1})^2}}{2}$ // Extra speed in last two steps\\
  }
  $\gamma_{k+1} \leftarrow \frac{1}{\beta} \frac{2\tau_k + \tau_{k+1}-1}{\tau_{k+1}}$ // Weight for the proximity operator\\
  $\mathbf{x}_{k+1} \leftarrow \mathbf{z}_k - \frac{1}{\beta} \nabla E(\mathbf{z}_k)$ // Move along the gradient\\
  $\mathbf{z}_{k+1} \leftarrow \mathbf{x}_{k+1} + \frac{\tau_{k} - 1}{\tau_{k+1}} (\mathbf{x}_{k+1} - x_k) + \frac{\tau_k}{\tau_{k+1}}(\mathbf{x}_{k+1} - \mathbf{y}_k) + \frac{\tau_k - 1}{\beta \gamma_k \tau_{k+1}}(\mathbf{z}_k - \mathbf{y}_k)$ // More advanced momentum rule\\
  $\mathbf{y}_{k+1} \leftarrow prox_{\gamma_{k+1}}(\mathbf{z}_{k+1})$\\
  $k \leftarrow k + 1$\\
 }
\end{algorithm}

\subsection{ADMM}
what is it and why is it good

\fi
\chapter{Related Works}

%\section{Parallel Imaging}
%"Self-Calibrating Nonlinear Reconstruction Algorithms for Variable Density Sampling and Parallel Reception MRI" by Loubna El Gueddari, C. Lazarus, H Carrié, A. Vignaud, Ph Ciuciu

%SENSE and ESPIRiT for sensitivity map estimation

\section{Proximal Optimized Gradient Method}
Why is it faster than FISTA, and why is it optimal

D. Kim and J. A. Fessler, “Optimized first-order methods for smooth convex minimization,” Math. Program., vol. 159, no. 1, pp. 81–107, Sep. 2016, doi: 10.1007/s10107-015-0949-3.

\section{Decompositions}

\subsection{Low rank and Sparse}
C. Y. Lin and J. A. Fessler, “Efficient Dynamic Parallel MRI Reconstruction for the Low-Rank Plus Sparse Model,” IEEE Transactions on Computational Imaging, vol. 5, no. 1, pp. 17–26, Mar. 2019, doi: 10.1109/TCI.2018.2882089.

J. A. Fessler, “Optimization methods for MR image reconstruction (long version),” arXiv:1903.03510 [eess, math], Jun. 2019.

\subsection{Multiscale}

Ong's dissertation: “Low Dimensional Methods for High Dimensional Magnetic Resonance Imaging,” 2018.

F. Ong et al., “Extreme MRI: Large-Scale Volumetric Dynamic Imaging from Continuous Non-Gated Acquisitions,” arXiv:1909.13482 [physics], Dec. 2019.

Differences between Ong's dissertation and his "extreme MRI" preprint paper

\section{IRSL}
Iteratively reweighted Least Squares method

C. Kümmerle and C. M. Verdun, “Denoising and Completion of Structured Low-Rank Matrices via Iteratively Reweighted Least Squares,” arXiv:1811.07472 [cs, math], Nov. 2018.

Henry Adams, Lara Kassab, and Deanna Needell "An Iterative Method for Structured Matrix Completion"

\clearpage % You need \clearpage at the end of every chapter to force images included in this chapter to be rendered in somewhere else
\chapter{Contributions}

While Julia produces usually faster code than other languages at the same abstraction level for numerical application, the true power of the language is only shown when the programmer writes the code following a couple performance tips listed on the official documentation. One of the most commonly occurring mistake that results suboptimal code is the high number of automatic memory allocations that leads to more frequent calls to garbage collector, and in other words, it wastes resources. Therefore in the following presentation of the programming-related contributions we will highlight the memory-efficiency at the evaluation of the created code base.

\section{FunctionOperators package}

As we have seen so far in the descriptions of the algorithms, operations very often are expressed as a multiplication a matrix-like entity and a vector or matrix. That notation is particularly convenient as it allows using the same set of mathematical tools as we have for matrices, while allows extension of the capabilities of the conventional linear mapping defined as a matrix-vector multiplication. In practice that means that we would like to combine the flexibility of functions with the power of tools designed for matrices, so many times the best option is to define the operation we want to perform as a function, and then wrap this function in an object that acts like a matrix and therefore it is compatible for instance with iterative solvers like the conjugate gradient method. Also it is desirable to add an "backward" operation, also defined by a function, to that wrapper object, that defines what should the solver do when it wants to use the adjoint of the operator.

Being driven by mostly scientists, the Julia community has already developed some packages that helps keeping the code structurally and visually similar to the abstract mathematical notation; nevertheless, none of these were completely satisfying. In particular, there are three relatively popular packages that have such functionality, but all of them some drawbacks:
\begin{itemize}
    \item LinearOperators.jl~\cite{noauthor_juliasmoothoptimizerslinearoperatorsjl_2020} and LinearMaps.jl~\cite{jutho_jutholinearmapsjl_2020} aims to provide almost all features of general matrices, but this design choice restricts the possible inputs to vectors.
    \item AbstractOperators.jl~\cite{noauthor_kul-forbesabstractoperatorsjl_2020} is a fairly new package that overcomes this limitation and provides a relatively large range of features, mostly in the form of predefined operators for the most common operations such as DFT, convolution, finite differences etc. It is also memory effective, preallocating a buffers when two operators are composed (that corresponds to the matrix-matrix multiplication) and it reuses this buffer later avoiding unnecessary memory allocations. Unfortunately, this optimization makes it impossible to accept on GPU arrays that makes unfavorable for large-scale applications.
\end{itemize}

As Julia is designed to be very extendable, it is always a feasible option to develop a new package that fits the specific problem. After reviewing the code of the mentioned packages, we came to the conclusion that the design choices of these packages doen't allow addition of features we desired, without breaking the already existing features, we decided to implement a package that fits better the image CS-MRI reconstruction framework. The developed package is named FunctionOperators.jl and it is already published to the central Julia package repository. Being just after the initial phase, the package supports only the most basic features, but the design of the interface and the efficient implementation lets the user build arbitrarily complex composite operators without having any computation penalty compared to the implementation with pure functions. As of today, the already implemented features if the package are the following:
\begin{itemize}
    \item Construction from a function with one argument that defines "forward" operation. When the constructed FunctionOperator is being multiplied with a vector/matrix of the \textit{proper} size, this function is called on the given vector/matrix. The size of the \textit{proper} input and expected output is a mandatory argument of the constructor of FunctionOperator, and any input with a mismatching size is rejected, and it after the multiplication the size of the output is also checked against the output size specified at construction. E.g. `Op = FunctionOperator(forw = x -> fft(x)); y = Op * x` calculates the FFT of x and stores in y.
    \item Construction from two functions both accepting one argument. These functions define the "forward" and the "backward" operations. In contrast to the "forward" function, the "backward" function is called when adjoint of the FunctionOperator is requested. E.g. `Op = FunctionOperator(forw = x -> fft(x), backw = x -> ifft(x)); y = Op' * x` calculates the inverse FFT of x and stores in y.
    \item These "forward" and "backward" functions also can accept two arguments. In that case, the second argument is the input of the operation, and the functions are expected to store the result of the operation in the first argument. This feature is advantageous if one wants to optimize the number of memory allocations. E.g. `Op = FunctionOperator(forw = (b,x) -> b .= x .* 2); mul!(y, Op, x)` multiplies x elementwise with two and stores the result in y without allocating an intermediate array.
    \item Composition of FunctionOperators by multiplication (that means composition of the "forward" functions), addition and substraction (these adds/substracts the output of the operations). E.g. considering `Op1 = FunctionOperator(forw = x -> fft(x)); Op2 = FunctionOperator(forw = (b,x) -> b .= x .* 2);` the output of `Op1 * Op2 * x` is the same as fft(x .* 2)
    \item Composition of FunctionOperators with UniformScaling object from LinearAlgebra standard library. This UniformScaling operator corresponds to the identity matrix. E.g. we can define the MR acquisition operator $\mathcal{A}$ as a FunctionOperator, and then we can define the CG operator in algorithm~\ref{alg:al-cg} as $\mathcal{A}' * \mathcal{A} + \delta_1 I$ (it is not the abstract mathematical notation, but a valid Julia expression!) and then we can pass it to a a solver (IterativeSolver.jl has, for example, CG solver that uses duck-typing, so it requires only that the multiplication must be implemented on the object passed as argument).
    \item Adjoint of composit FunctionOperators: If the FunctionOperator is a composition of other FunctionOperators, then the adjoint is defined as the composition of adjoints of the member FunctionOperators, in the reverse order. E.g., let us recall the decomposition of the acquisition operator $\mathcal{A} = \Omega \mathcal{FC}$. Now look on it on the other way around: we have $\Omega, \mathcal{F}$ and $\mathcal{C}$ already created as FunctionOperators, and we create the acquisition operator as the composition of them: $\mathcal{A} = \Omega * \mathcal{F * C}$. Then $\mathbf{A}' * x$ would do the same as $\mathcal{C' * F'} * \Omega' * x$.
\end{itemize}

Under the hood, the most important property of the package that it uses the least amount of memory possible by deferring the allocation of buffers as much as possible and also by maintaining a smart global pool that stores the already allocated buffers. These buffers are only needed to store intermediate results, and therefore they are safe to reuse in different FunctionOperators provided that they are on the same thread. But the user don't need to worry about that criterion since this case is handled automatically, making FunctionOperators a thread-safe package (assuming that the user builds the operators from thread-safe "forward" and "backward" functions). This memory effective implementation makes FunctionOperators a truly unique as other packages are all more or less suboptimal in this sense.

As an extra (yet experimental) feature, FunctionOperators package provides a macro that automatically optimizes loops by cutting down the number unnecessary memory allocations. To achieve this the macro uses the advanced code generation features that produces code before the compilation of the program, but after the parser has built the abstract syntax tree and deduced the type of variables; therefore, heavy optimizations can be performed behind the scenes using macros and generated functions. %Fig.~\ref{} shows three versions of the same code fragment: one is a readable version that resembles the mathematical formulation, the other is a memory-wise optimal, but much less readable code, and the third one show the usage of the macro that reduces the amount of allocated memory by $~70\%$.

Following the guidelines of the Julia community for package development, the code is thoroughly tested using unit tests, and has a clean documentation that contains a notebook with an example for almost all features. The coverage of unit tests are measured by \url{codecov.io}, and it reported that the $94\%$ of the code is covered. This report, however, underestimates the coverage as it fails to detect the covered lines in case of tests checking the functions that prints to console. The code of the package is uploaded to \url{https://github.com/hakkelt/FunctionOperators.jl}, and the documentation is available at \url{https://hakkelt.github.io/FunctionOperators.jl/latest/}.

\section{Implementation of Sparse+Low Rank algorithms}

After the careful examination of both the publication and the reference Matlab implementation, we created an efficient Julia version for each of these algorithms. As the implementation was done at the same time as the FunctionOperators was developed, it was a natural choice to benchmark the newly developed package against the other similar packages on these algorithms. The contribution of this part of the project is two-fold: First, it extends the currently small code base of MRI-related algorithms in Julia, helping the fast growing group of scientists choosing to prototype their research algorithms in Julia. Second, it might also help those who needs guidance in picking the right package, especially because currently no other comparison is available to see the difference between LinearMaps.jl and AbstractOperators.jl.

The result of the benchmarking is available in table~\ref{tab:lin_fessler} and the Jupyter notebooks holding both the benchmarking code, the documentation, and the results are available at \url{https://github.com/hakkelt/reproduce-l-s-dynamic-mri-julia}
In comparison to the Matlab implementation, the results are somewhat mixed since Julia outperformed Matlab in case of the improved AL scheme (the speedup here was ~2x) and in case of proximal methods for the non-Cartesian dataset (with 2-3.5x speedup). On the other hand, Matlab produced faster code for the other cases. This results underlines the fact that merely switching the language to Julia not necessarily results in increased speed automatically. A possible explanation is that we missed some hidden optimization tricks deeply buried in the Matlab code (which was well optimized indeed, and very hard to read---it required a fair amount of time from us to understand how is it connected to the theory described in the paper). Another possible factor is that Matlab uses some proprietary C and fortran libraries that have slightly better performance compared to the open source options bundled with Julia by default.

Furthermore, the comparison of different Julia implementations revealed that the three package have very similar running time and they differ mostly in the memory allocated, LinearMaps.jl having the largest memory demand (as it was expected from the implementation that allocates and releases buffers in each iteration again and again in our case), and FunctionOperators being better by a small margin. We also tested the automatic optimization macro mentioned above, and the benchmarks showed that it reached almost optimal memory usage, reducing the size of net allocations by $~70\%$ on average, compared to the $~80\%$ reduction achieved by the tedious process of manual optimization.

\begin{table}[]
\footnotesize
\begin{tabular}{|p{0.1\linewidth}|p{0.18\linewidth}p{0.18\linewidth}p{0.18\linewidth}p{0.18\linewidth}|}
\hline
algorithm \textbackslash data set & PINCAT & Multicoil cardiac cine MRI & Multicoil cardiac perfusion MRI & Multicoil abdominal dce MRI \\ \hline
\multicolumn{1}{|c|}{} & \multicolumn{4}{c|}{Matlab} \\
AL-CG & 16.8 s & 49.0 s & 19.9 s & - \\
AL-2 & 17.8 s & 55.3 s & 27.5 s & - \\
ISTA & 1.7 s & 5.5 s & 2.3 s & 141.8 s \\
FISTA & 2.4 s & 7.6 s & 3.1 s & 256.7 s \\
POGM & 1.8 s & 5.8 s & 2.3 s & 140.0 s \\ \hline
\multicolumn{1}{|c|}{} & \multicolumn{4}{c|}{LinearMaps} \\
AL-CG & 28.4 s, 11.26 GiB & 80.8 s, 31.08 GiB & 33.8 s, 12.95 GiB & - \\
AL-2 & 9.8 s, 6.22 GiB & 28.4 s, 17.56 GiB & 11.0 s, 7.32 GiB & - \\
ISTA & 6.0 s, 627.71 MiB & 19.6 s, 1.36 GiB & 6.8 s, 582.06 MiB & 72.7 s, 2.18 GiB \\
FISTA & 6.2 s, 627.71 MiB & 16.5 s, 1.36 GiB & 6.8 s, 582.06 MiB & 73.2 s, 2.18 GiB \\
POGM & 6.3 s, 677.71 MiB & 16.8 s, 1.45 GiB & 6.8 s, 622.06 MiB & 72.6 s, 2.42 GiB \\ \hline
\multicolumn{1}{|c|}{} & \multicolumn{4}{c|}{AbstractOperators} \\
AL-CG & 28.7 s, 678.06 MiB & 76.1 s, 1.45 GiB & 32.1 s, 622.41 MiB & - \\
AL-2 & 9.0 s, 1.14 GiB & 23.3 s, 2.93 GiB & 10.2 s, 1.22 GiB & - \\
ISTA & 6.4 s, 627.68 MiB & 16.4 s, 1.36 GiB & 7.3 s, 582.03 MiB & 72.8 s, 2.18 GiB \\
FISTA & 6.5 s, 627.68 MiB & 16.3 s, 1.36 GiB & 7.3 s, 582.03 MiB & 72.0 s, 2.18 GiB \\
POGM & 6.6 s, 677.68 MiB & 16.5 s, 1.45 GiB & 7.0 s, 622.03 MiB & 73.9 s, 2.42 GiB \\ \hline
\multicolumn{1}{|c|}{} & \multicolumn{4}{c|}{FunctionOperators naive} \\
AL-CG & 30.6 s, 2.90 GiB & 79.4 s, 6.14 GiB & 33.3 s, 2.43 GiB & - \\
AL-2 & 11.6 s, 12.20 GiB & 32.0 s, 33.17 GiB & 13.1 s, 13.82 GiB & - \\
ISTA & 6.8 s, 3.40 GiB & 18.1 s, 8.67 GiB & 7.6 s, 3.62 GiB & 72.7 s, 6.71 GiB \\
FISTA & 6.9 s, 3.40 GiB & 17.6 s, 8.67 GiB & 7.5 s, 3.62 GiB & 74.9 s, 6.71 GiB \\
POGM & 6.9 s, 3.45 GiB & 17.8 s, 8.77 GiB & 7.5 s, 3.65 GiB & 73.741 s, 6.96 GiB \\ \hline
\multicolumn{1}{|c|}{} & \multicolumn{4}{c|}{functionOperators optimized} \\
AL-CG & 27.0 s, 640.69 MiB & 76.2 s, 1.38 GiB & 33.0 s, 592.54 MiB & - \\
AL-2 & 8.6 s, 1.04 GiB & 22.7 s, 2.65 GiB & 10.0 s, 1.11 GiB & - \\
ISTA & 6.1 s, 627.79 MiB & 16.6 s, 1.36 GiB & 7.6 s, 582.14 MiB & 75.1 s, 2.18 GiB \\
FISTA & 6.1 s, 627.79 MiB & 16.5 s, 1.36 GiB & 7.9 s, 582.14 MiB & 73.8 s, 2.18 GiB \\
POGM & 6.3 s, 677.79 MiB & 16.747 s, 1.45 GiB & 7.079 s, 622.14 MiB & 73.6 s, 2.42 GiB \\ \hline
\multicolumn{1}{|c|}{} & \multicolumn{4}{c|}{FunctionOperators pretty} \\
AL-CG & 28.0 s, 741.02 MiB & 75.1 s, 1.57 GiB & 30.7 s, 662.87 MiB & - \\
AL-2 & 8.4 s, 1.26 GiB & 21.8 s, 3.26 GiB & 9.3 s, 1.36 GiB & - \\
ISTA & 6.3 s, 1.05 GiB & 16.4 s, 2.49 GiB & 6.9 s, 1.04 GiB & 73.5 s, 3.03 GiB \\
FISTA & 6.3 s, 1.05 GiB & 16.3 s, 2.49 GiB & 6.9 s, 1.04 GiB & 72.2 s, 3.03 GiB \\
POGM & 6.4 s, 1.10 GiB & 16.5 s, 2.58 GiB & 6.9 s, 1.08 GiB & 74.5 s, 3.28 GiB \\ \hline
\end{tabular}
\caption{Benchmarking on different datasets performing reconstruction via proximity and augmented Lagrangian methods descirbed in~\cite{lin_efficient_2019}. "FunctionOperators naive" is an implementation with minimal manual optimizations, "FunctionOperators optimized" is a manually optimized version, and ""FunctionOperators pretty" is exactly same as the "naive" version except that our automatic automatic optimizer macro is called on the code.}
\label{tab:lin_fessler}
\end{table}

\section{Implementation of Multiscale Decomposition}

\subsection{NUFFT}

\subsection{MSLR Algorithm}

%\subsection{Optimization possibilities}
%\begin{enumerate}
%    \item Batch-processing: compute NUFFT of all channels at once
%    \item Parallelization:
%    \begin{enumerate}
%        \item NUFFT
%        \item Algorithm
%    \end{enumerate}
%    \item GPU-specific optimizations:
%    \begin{enumerate}
%        \item Blocking operator might cause CPU bottleneck
%    \end{enumerate}
%\end{enumerate}

%\subsection{Parallelization Solution}
%Details how I made the code run parallel

\begin{figure}
    \centering
    \begin{minipage}{0.48\linewidth}
        \centering
        \includegraphics[width=\linewidth]{images/nfft_small_forw.pdf}
        \label{fig:nfft_small_forw}
    \end{minipage}
    \begin{minipage}{0.48\linewidth}
        \centering
        \includegraphics[width=\linewidth]{images/nfft_small_backw.pdf}
        \label{fig:nfft_small_backw}
    \end{minipage}
    \caption{\textbf{Comparison of running time} between the Python implementation in SigPy package (blue), and the Julia implementations (orange: initialization and computation, green: computation only) at different threading settings. (Note that the Python implementation does not implement multithreading, and hence the height of the three blue bar is the same.) \textbf{For small images, single threaded Julia version is the fastest.}}
    \label{fig:nfft_small}
\end{figure}


\begin{figure}
    \centering
    \begin{minipage}{0.48\linewidth}
        \centering
        \includegraphics[width=\linewidth]{images/nfft_medium_forw.pdf}
        \label{fig:nfft_medium_forw}
    \end{minipage}
    \begin{minipage}{0.48\linewidth}
        \centering
        \includegraphics[width=\linewidth]{images/nfft_medium_backw.pdf}
        \label{fig:nfft_medium_backw}
    \end{minipage}
    \caption{\textbf{Comparison of running time} between the Python implementation in SigPy package (blue), and the Julia implementations (orange: initialization and computation, green: computation only) at different threading settings. (Note that the Python implementation does not implement multithreading, and hence the height of the three blue bar is the same.) \textbf{For small images, single threaded Julia version is the fastest.}}
    \label{fig:nfft_medium}
\end{figure}

\begin{figure}
    \centering
    \begin{minipage}{0.48\linewidth}
        \centering
        \includegraphics[width=\linewidth]{images/nfft_large_forw.pdf}
        \label{fig:nfft_large_forw}
    \end{minipage}
    \begin{minipage}{0.48\linewidth}
        \centering
        \includegraphics[width=\linewidth]{images/nfft_large_backw.pdf}
        \label{fig:nfft_large_backw}
    \end{minipage}
    \caption{\textbf{Comparison of running time} between the Python implementation in SigPy package (blue), and the Julia implementations (orange: initialization and computation, green: computation only) at different threading settings. (Note that the Python implementation does not implement multithreading, and hence the height of the three blue bar is the same.) \textbf{For small images, single threaded Julia version is the fastest.}}
    \label{fig:nfft_large}
\end{figure}

\begin{figure}
    \centering
    \begin{minipage}{0.48\linewidth}
        \centering
        \includegraphics[width=\linewidth]{images/gridding_recon_speed.pdf}
        \label{fig:gridding_recon_speed}
    \end{minipage}
    \begin{minipage}{0.48\linewidth}
        \centering
        \includegraphics[width=\linewidth]{images/MSLR_recon_speed.pdf}
        \label{fig:MSLR_recon_speed}
    \end{minipage}
    \caption{asdf}
    \label{fig:3D_recon}
\end{figure}

\section{IRLS Implementation}

%\subsection{Description of Algorithm}
%\subsection{Implementation Details}

\begin{figure}
    \centering
    \begin{minipage}{0.48\linewidth}
        \centering
        \includegraphics[width=\linewidth]{images/ideal_MSE.pdf}
        \label{fig:ideal_MSE}
    \end{minipage}
    \begin{minipage}{0.48\linewidth}
        \centering
        \includegraphics[width=\linewidth]{images/ideal_rank.pdf}
        \label{fig:ideal_rank}
    \end{minipage}
    \caption{asdf}
    \label{fig:ideal_recon}
\end{figure}

\begin{figure}
    \centering
    \begin{minipage}{0.48\linewidth}
        \centering
        \includegraphics[width=\linewidth]{images/reconstruction_power.pdf}
        \label{fig:reconstruction_power}
    \end{minipage}
    \begin{minipage}{0.48\linewidth}
        \centering
        \includegraphics[width=\linewidth]{images/noise_tolerance.pdf}
        \label{fig:noise_tolerance}
    \end{minipage}
    \caption{asdf}
    \label{fig:IRLS_analysis}
\end{figure}

\begin{figure}
    \centering
    \begin{minipage}{0.48\linewidth}
        \centering
        \includegraphics[width=\linewidth]{images/orig_MSE.pdf}
        \label{fig:orig_MSE}
    \end{minipage}
    \begin{minipage}{0.48\linewidth}
        \centering
        \includegraphics[width=\linewidth]{images/orig_rank.pdf}
        \label{fig:orig_rank}
    \end{minipage}
    \caption{asdf}
    \label{fig:orig}
\end{figure}

\begin{figure}
    \centering
    \begin{minipage}{0.48\linewidth}
        \centering
        \includegraphics[width=\linewidth]{images/orig_MSE.pdf}
        \label{fig:orig_MSE}
    \end{minipage}
    \begin{minipage}{0.48\linewidth}
        \centering
        \includegraphics[width=\linewidth]{images/orig_rank.pdf}
        \label{fig:orig_rank}
    \end{minipage}
    \caption{asdf}
    \label{fig:orig}
\end{figure}

\begin{figure}
    \centering
    \includegraphics[width=0.46\linewidth]{images/PINCAT_all.pdf}
    \caption{Caption}
    \label{fig:PINCAT_all}
\end{figure}

\begin{figure}
    \centering
    \includegraphics[width=0.46\linewidth]{images/PINCAT_diff_t20.pdf}
    \caption{Caption}
    \label{fig:PINCAT_diff_t20}
\end{figure}

\begin{figure}
    \centering
    \includegraphics[width=0.46\linewidth]{images/PINCAT_IRLS_recon.pdf}
    \caption{Caption}
    \label{fig:PINCAT_IRLS_recon}
\end{figure}

\begin{figure}
    \centering
    \includegraphics[width=0.46\linewidth]{images/PINCAT_IRLS_recon_error.pdf}
    \caption{Caption}
    \label{fig:PINCAT_IRLS_recon_error}
\end{figure}

\begin{figure}
    \centering
    \includegraphics[width=0.46\linewidth]{images/PINCAT_MSLR_recon.pdf}
    \caption{Caption}
    \label{fig:PINCAT_MSLR_recon}
\end{figure}

\begin{figure}
    \centering
    \includegraphics[width=0.46\linewidth]{images/PINCAT_POGM_recon.pdf}
    \caption{Caption}
    \label{fig:PINCAT_POGM_recon}
\end{figure}

\clearpage % You need \clearpage at the end of every chapter to force images included in this chapter to be rendered in somewhere else

%\chapter{Results}
% The Code of Studies and Exams recommends the following content for this chapter: "Evaluation, critical analysis of the implemented technical solutions, possibilities for further development."
% However, you are free to structure the content of your thesis as you want

\section{Sparse + Low Rank Algorithm}
\subsection{Running Speed and Memory Used}
\subsection{Readability}

\section{Multi-scale Algorithm}
\subsection{Running Speed and Memory Used}

\section{ILRS}
\subsection{Comparism with Sparse+Low Rank}
\paragraph{Running speed and memory requirement}
\paragraph{Convergence speed}
\paragraph{Robustness:} Effect of error on the input data
\paragraph{Recovery capability:} Can we reach the same accuracy on recovered image using less data?

\subsection{Comparism with Multi-scale}
\paragraph{Running speed and memory requirement}
\paragraph{Convergence speed}
\paragraph{Robustness:} Effect of error on the input data
\paragraph{Recovery capability:} Can we reach the same accuracy on recovered image using less data?

\clearpage % You need \clearpage at the end of every chapter to force images included in this chapter to be rendered in somewhere else
\chapter{Summary}
% The Code of Studies and Exams recommends the following content for this chapter:
% A summary of the problems solved compared to the objectives presented in the introduction and  in the Thesis Proposal Form. Opportunities to move forward, questions motivating the future works, outlook.
% However, you are free to structure the content of your thesis as you want

\section{Objectives}
\section{Achievements}
\section{Future Plans}

% Bibliography
\printbibliography
\addcontentsline{toc}{chapter}{Bibliography}

% Appendices -- if you don't have any, just delete the following two lines
%\appendix
%\chapter{Appendix}

Here you can present all the materials that helps the understanding of your work, but either 1) not your work, 2) not necessary to understanding, 3) simply just interesting things not directly related to your topic, or 4) your text already exceeded 120 pages, and need to make it shorter by moving parts to appendix... :D Anyway, if you have a large amount of images, code, or measurement data, it is required to insert them here rather than in chapters. You can have multiple appendices (possible separated into multiple files) according to the content to be attached.


\end{document}

